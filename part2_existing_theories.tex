% ============================================================================
% PART II: BRIDGING TO EXISTING THEORIES
% ============================================================================

\part{Bridging to Existing Theories}

% ============================================================================
% CHAPTER 3: INTEGRATED INFORMATION THEORY
% ============================================================================

\chapter{Integrated Information Theory Through the Lens of Finite Machines}

\section{IIT Foundations and Core Concepts}

Integrated Information Theory (IIT), developed primarily by Giulio Tononi and colleagues \autocite{tononi2004,tononi2008,tononi2016}, proposes that consciousness corresponds to integrated information, quantified as $\Phi$ (phi). The theory begins with phenomenological axioms about conscious experience and derives postulates about the physical substrates that can support consciousness \autocite{oizumi2014}.

IIT's five axioms capture essential properties of experience: intrinsic existence (consciousness exists from its own perspective), composition (consciousness is structured), information (each experience is specific), integration (experience is unified), and exclusion (experience is definite). These phenomenological axioms translate into physical postulates about systems that can be conscious.

The central quantity $\Phi$ measures integrated information—the amount of information generated by a system above and beyond what its parts generate independently. High $\Phi$ systems have strongly integrated cause-effect structures where partitioning significantly degrades information. IIT identifies consciousness with $\Phi^{\text{max}}$, the maximum of integrated information over all possible spatial and temporal scales.

\subsection{How Our Framework Relates to IIT}

Our framework doesn't contradict IIT but provides a deeper computational explanation. We propose that $\Phi$ measures integration within a machine level $M_n$, not consciousness itself. High $\Phi$ is necessary but insufficient for consciousness. The cerebellum has high $\Phi$ but lacks consciousness because it implements only low-level machines without the hierarchical structure and selector mechanism needed for conscious experience.

Consciousness arises not from $\Phi$ alone but from the collapse process selecting among hierarchically organized machines. Within the selected machine, high integration ($\Phi$) is essential—consciousness requires integrated information processing. But the integration must occur within a machine capable of being selected by the non-computable selector mechanism operating across the hierarchy.

This explains IIT's empirical successes while addressing its challenges. IIT correctly identifies integration as crucial for consciousness. Its predictions about neural correlates of consciousness align with our framework—consciousness requires integrated thalamocortical processing spanning multiple regions. Where IIT struggles with computational intractability, our framework suggests this reflects the fundamental non-computability of the selector, not merely a technical limitation.

\section{Global Workspace Theory}

\subsection{GWT Foundations}

Global Workspace Theory (GWT), developed by Bernard Baars \autocite{baars1988,baars1997} and extended by Stanislas Dehaene \autocite{dehaene2001}, proposes that consciousness acts as a broadcast mechanism making information globally available to specialized cognitive processes. The theory uses a theater metaphor: consciousness is like a spotlight on a stage, bringing some information into global availability while leaving the rest in darkness.

GWT emphasizes the limited capacity of consciousness—only a small amount of information can be globally broadcast at once—and the wide distribution of unconscious processing. Consciousness enables coordination among specialized processors by making selected information available to all. This explains attention, working memory limitations, and the relationship between consciousness and report.

\subsection{Integration with Our Framework}

Our framework explains what GWT describes. Global availability is the result of collapse, not its cause. The sequence is: parallel exploration across machines → selector-driven collapse to one path → global availability of the collapsed state. GWT identifies the final step (broadcast) with consciousness, while our framework shows that consciousness is the collapse process, with broadcast being its consequence.

This explains several puzzles for GWT. Why should global availability produce phenomenology rather than remaining unconscious? Because the collapse that makes information globally available is itself consciousness—the broadcast doesn't create experience but expresses the already-conscious collapsed state. Why is attention capacity limited? Because the selector can only deploy one machine level for focal processing at once. Why do some globally broadcast signals remain unconscious while others are conscious? Because global availability alone isn't sufficient—the information must be part of the collapsed path.

The theater metaphor needs reinterpretation: the spotlight (attention) reflects selector operation, the stage (global workspace) represents the space of possible collapses, and consciousness is not the illuminated content but the process of selecting what to illuminate. GWT's neural predictions—frontal-parietal activation, ignition dynamics, broadcast signatures—all correspond to post-collapse processes in our framework.

% ============================================================================
% CHAPTER 4: ATTENTION SCHEMA THEORY
% ============================================================================

\chapter{Attention Schema Theory and Other Approaches}

\section{AST Foundations}

Attention Schema Theory (AST), developed by Michael Graziano \autocite{graziano2013,graziano2019}, proposes that consciousness is the brain's internal model of its own attention. Just as the brain builds body schemas to model the body's state, it builds an attention schema to model the deployment and dynamics of attention. This schema serves control and predictive functions, allowing the brain to manipulate and predict attention efficiently.

AST explains consciousness as a control model—a simplified, compressed representation used for guiding attention. The schema attributes properties to attention (location, intensity, clarity) that allow efficient control without requiring detailed mechanism knowledge. Consciousness is what the attention schema represents, not attention itself. This is why consciousness feels like a unified, directed property—the schema models it that way for control purposes.

\subsection{Integration with Our Framework}

The attention schema in AST corresponds to the brain's model of selector operation in our framework. When you're conscious of attending to something, the attention schema is representing the selector's deployment of resources to that content. The schema doesn't represent detailed machine hierarchy architecture but provides a simplified model sufficient for voluntary control.

This explains AST's insights while addressing its limitations. AST asks why there should be phenomenology associated with the schema—why shouldn't the brain model attention unconsciously? Our answer: because the selector's operation is the collapse process that constitutes consciousness. The attention schema models this collapse, making metacognitive access to conscious state possible. The schema itself can become an object of further collapse, creating consciousness of consciousness.

The relationship between attention and consciousness becomes clear: attention is selector operation (resource deployment), consciousness is collapse (selection of one path), and the attention schema models both. Not all attention is conscious (automatic resource deployment without collapse), and not all consciousness requires focal attention (peripheral awareness involves different machine levels), but prototypical conscious experience involves both selector-driven attention and schema-based metacognition.

\section{Other Theories Briefly Considered}

\subsection{Higher-Order Theories}

Higher-order theories propose that consciousness requires higher-order representations—thoughts about mental states or representations of representations. These theories capture something important: metacognition is central to human consciousness. Our framework explains this through the machine hierarchy: higher machines can represent the states of lower machines, creating hierarchical representations. Consciousness doesn't require higher-order representation but sophisticated consciousness (the kind humans have) benefits from it.

\subsection{Quantum Consciousness}

Quantum consciousness theories \autocite{penrose1989,hameroff1996} propose that quantum processes in neural microtubules create consciousness through objective reduction. While our framework doesn't require quantum effects, it's compatible with them. If quantum non-determinism influences selector operation, this could provide a physical implementation of the selector's non-computability. However, the framework's core claims about machine hierarchies, selection, and collapse don't depend on quantum mechanics—they're computational principles that could be implemented classically or quantum-mechanically.

\subsection{Predictive Processing}

Predictive processing frameworks propose that the brain constantly predicts sensory input and updates predictions based on prediction errors. This connects naturally to our framework: parallel exploration generates predictions, collapse selects the best-predicting model, and the collapsed state determines conscious experience. Prediction error drives selector operation—large errors trigger resource escalation (deploying higher machines), while accurate predictions allow resource de-escalation.

\section{Synthesis: A Meta-Theory}

Our framework functions as a meta-theory that explains what existing theories have discovered. IIT identifies integration as necessary within machine levels. GWT describes the broadcast that follows collapse. AST models the selector's operation. Higher-order theories describe metacognitive representations across hierarchy levels. Predictive processing characterizes how parallel exploration operates. Each theory captures one aspect of the complete computational architecture.

This isn't eclecticism but synthesis. The machine hierarchy with selector and collapse provides the unifying structure that explains why these diverse theories each succeeded in their domains while failing to provide complete accounts. Consciousness requires integration (IIT), produces global availability (GWT), involves attention schema (AST), enables higher-order representation (HOT), and implements predictive processing. Our framework shows how these fit together into a coherent computational whole.

% ============================================================================
% CHAPTER 5: COMPARATIVE ANALYSIS
% ============================================================================

\chapter{Comparative Analysis: Strengths and Limitations}

\section{What Each Theory Explains Well}

IIT excels at characterizing the information-theoretic properties of conscious systems, explaining why certain brain regions (thalamocortical system) support consciousness while others (cerebellum) don't despite neural complexity. GWT explains attention, working memory, and the relationship between consciousness and report, capturing the functional architecture of access consciousness. AST provides the most compelling account of metacognition and voluntary control, explaining the phenomenology of directed attention. Our framework explains these successes while addressing their limitations.

\section{What Each Theory Struggles With}

IIT struggles with computational intractability (computing $\Phi$ is prohibitively difficult), counterintuitive implications (simple systems can have arbitrarily high $\Phi$), and the hard problem (why should $\Phi$ feel like anything?). GWT struggles with phenomenology (why should broadcast be conscious?), explaining qualia, and accounting for unconscious-but-globally-available information. AST struggles with the hard problem (why should a model feel like anything?) and explaining the qualitative character of experience beyond functional description.

Our framework addresses these limitations by identifying consciousness with computational collapse across hierarchical machines. This explains phenomenology (collapse is intrinsically experiential), handles the hard problem (dissolves rather than solves it), accounts for both integration and broadcast (both are aspects of collapse), and explains metacognition (schema models selector operation).

\section{Empirical Predictions and Distinguishability}

Each theory makes specific predictions that can be empirically tested. IIT predicts that $\Phi$ correlates with consciousness, GWT predicts frontal-parietal broadcast signatures, AST predicts that disrupting the attention schema disrupts consciousness. Our framework makes additional predictions: machine hierarchy organization, parallel exploration signatures, collapse dynamics, selector non-computability.

Critically, our predictions differ from existing theories in testable ways. We predict that $\Phi$ measures integration within levels, not consciousness across levels. We predict that broadcast follows collapse rather than creating it. We predict that the attention schema models selector operation, making schema accuracy and selector efficiency related but distinct. These differences enable empirical discrimination between theories.

\section{The Path Forward}

The field of consciousness science needs both theoretical diversity and synthetic integration. Existing theories each captured important insights about consciousness. Our framework attempts synthesis not by rejecting these insights but by showing how they fit into a comprehensive computational architecture. Future work should test whether this synthesis succeeds empirically, whether it generates novel predictions, and whether it provides a more complete explanation than existing theories alone.
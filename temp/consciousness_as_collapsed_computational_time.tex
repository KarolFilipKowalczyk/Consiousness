\documentclass[11pt,a4paper]{article}
\usepackage[utf8]{inputenc}
\usepackage[T1]{fontenc}
\usepackage{amsmath,amssymb,amsthm}
\usepackage{mathtools}
\usepackage{geometry}
\usepackage{hyperref}
\usepackage{enumitem}

\geometry{margin=1in}

% Theorem environments
\newtheorem{theorem}{Theorem}[section]
\newtheorem{lemma}[theorem]{Lemma}
\newtheorem{proposition}[theorem]{Proposition}
\newtheorem{corollary}[theorem]{Corollary}
\theoremstyle{definition}
\newtheorem{definition}[theorem]{Definition}
\newtheorem{example}[theorem]{Example}
\theoremstyle{remark}
\newtheorem{remark}[theorem]{Remark}

% Custom commands for entropic scaling
\newcommand{\InfoCapacity}[1]{I(#1)}
\newcommand{\EntropicScale}[1]{\kappa #1 \log #1}

% Global entropic scaling directive:
% All hierarchical quantities use I(n) = κ n log n instead of 2^n

\title{Consciousness as Collapsed Computational Time:\\
A Unified Framework with Entropic Scaling}
\author{Karol Kowalczyk}
\date{\today}

\begin{document}

\maketitle

\begin{abstract}
We present a unified theory of consciousness based on the collapse of parallel computational explorations across a hierarchy of finite-state machines with entropic information scaling. Each machine $M_n$ has effective information capacity $I(n) = \kappa n \log n$ bits, providing super-linear but sub-exponential growth that avoids unrealistic resource requirements. Consciousness emerges when parallel explorations of computational paths collapse into a single experienced trajectory, with the collapse process requiring time $\tau(n) = \tau_0 + \gamma n \log n$. This entropic temporal phenomenology explains why subjective experience feels continuous despite the underlying computational discreteness. The framework unifies insights from integrated information theory, global workspace theory, and quantum approaches while making specific testable predictions about neural signatures and temporal dynamics.
\end{abstract}

\section{Introduction}

The hard problem of consciousness asks why there is "something it is like" to be an information-processing system. We propose that consciousness is the internal phenomenology of computational collapse in a hierarchically organized system with entropic resource scaling. Rather than requiring exponential resources, our framework operates with realistic information capacities that scale as $I(n) = \kappa n \log n$.

\section{The Machine Hierarchy}

\subsection{Entropic Information Scaling}

\begin{definition}[Machine Hierarchy with Entropic Scaling]
The consciousness substrate consists of a hierarchy of finite-state machines $\{M_n\}_{n=1}^{n_{\max}}$ where each machine $M_n$ has:
\begin{itemize}
\item Effective information capacity: $I(n) = \kappa n \log n$ bits
\item State space with approximately $\exp(I(n))$ distinguishable states
\item Transition function $f_n: S_n \to S_n$
\item Processing time: $\tau(n) = \tau_0 + \gamma n \log n$
\end{itemize}
\end{definition}

The entropic scaling $I(n) = \kappa n \log n$ captures the realistic growth of information processing capacity in biological systems. Unlike exponential scaling ($2^n$), this provides:
\begin{itemize}
\item Super-linear growth ensuring increased capacity with level
\item Sub-exponential bounds preventing resource explosion
\item Alignment with thermodynamic entropy and information theory
\item Biologically plausible resource requirements
\end{itemize}

\subsection{Hierarchy Properties}

\begin{proposition}[Strict Inclusion with Entropic Scaling]
The hierarchy exhibits strict inclusion: for $i < j$,
\begin{equation}
M_i \subset M_j \quad \text{with} \quad I(j) - I(i) = \kappa(j \log j - i \log i)
\end{equation}
This ensures higher levels can solve strictly more problems while maintaining realistic resource gaps.
\end{proposition}

\section{Temporal Collapse Dynamics}

\subsection{Computational vs. Subjective Time}

The framework distinguishes two temporal experiences:

\begin{definition}[Temporal Duality]
\begin{align}
t_{\text{comp}} &= \text{Full computational time including all explorations} \\
t_{\text{subj}} &= \text{Subjective time experiencing only the collapsed path}
\end{align}
With entropic scaling, the collapse time is:
\begin{equation}
\tau_{\text{collapse}}(n) = \tau_0 + \gamma n \log n
\end{equation}
\end{definition}

\subsection{Proper Time Expression}

The infinitesimal proper time in the collapsed state follows:
\begin{equation}
d\tau_c = \phi(T,\rho) \cdot dI = \phi(T,\rho) \cdot \kappa(\log n + 1) \, dn
\end{equation}
where $\phi(T,\rho)$ is a phenomenological function of temperature and density.

\subsection{Convergence Windows}

\begin{theorem}[Entropic Convergence Windows]
The time window for parallel explorations to converge scales as:
\begin{equation}
\Delta \tau_c \propto n \log n
\end{equation}
This entropic scaling ensures convergence occurs within biologically relevant timescales (100-500ms for typical cognitive levels).
\end{theorem}

\section{The Selector Mechanism}

\subsection{Resource Optimization}

The selector $\mathcal{S}$ chooses which machine level to deploy:

\begin{definition}[Entropic Selector]
The selector minimizes total cost:
\begin{equation}
\mathcal{S}(p) = \arg\min_n \left[ K_n(p) + C(n) \right]
\end{equation}
where $K_n(p)$ is the Kolmogorov complexity of problem $p$ at level $n$, and the cost function follows entropic scaling:
\begin{equation}
C(n) = C_0(1 + \beta n \log n)
\end{equation}
\end{definition}

\subsection{Non-computability and Agency}

\begin{theorem}[Selector Non-computability]
No algorithm can compute the optimal selector function $\mathcal{S}^*$ for all problems. This non-computability, combined with entropic resource constraints, generates genuine agency.
\end{theorem}

The proof follows from the uncomputability of Kolmogorov complexity, with the entropic scaling ensuring the search space remains tractable enough for heuristic approximation.

\section{Integration and Phenomenology}

\subsection{Integrated Information with Entropic Normalization}

\begin{definition}[Entropic Integrated Information]
For a system at level $n$ with information capacity $I(n) = \kappa n \log n$:
\begin{equation}
\Phi_n = \min_{\text{partition}} \left[ I(W; W') - \sum_i I(W_i; W_i') \right] \cdot \frac{1}{n \log n}
\end{equation}
The normalization by $1/(n \log n)$ ensures $\Phi$ remains scale-invariant across levels.
\end{definition}

\subsection{Phenomenological Unity}

The entropic scaling ensures that integration creates unified experience without requiring exponential resources:

\begin{proposition}[Unity Condition]
Consciousness requires $\Phi_n > \Phi_{\text{threshold}}$ where the threshold depends on the entropic information capacity rather than raw state count.
\end{proposition}

\section{Testable Predictions}

\subsection{Neural Signatures}

The entropic scaling predicts specific neural signatures:

\begin{enumerate}
\item \textbf{Discrete capacity jumps}: Working memory capacity should show discrete levels corresponding to $I(n) = \kappa n \log n$ for integer $n$
\item \textbf{Temporal scaling}: Neural processing time should follow $\tau(n) = \tau_0 + \gamma n \log n$
\item \textbf{Integration measures}: Phi should scale with entropic normalization $1/(n \log n)$
\end{enumerate}

\subsection{Behavioral Predictions}

\begin{enumerate}
\item \textbf{Reaction times}: Should cluster at values predicted by $\tau(n) = \tau_0 + \gamma n \log n$
\item \textbf{Capacity limits}: Cognitive capacity should follow entropic rather than linear or exponential scaling
\item \textbf{Transition dynamics}: Level transitions should show signatures of $n \log n$ complexity
\end{enumerate}

\section{Comparison with Existing Theories}

\subsection{Advantages of Entropic Scaling}

\begin{center}
\begin{tabular}{lcc}
\hline
\textbf{Theory} & \textbf{Resource Scaling} & \textbf{Biological Plausibility} \\
\hline
IIT (Original) & Exponential ($2^n$) & Low \\
GWT & Unspecified & Medium \\
Predictive Processing & Hierarchical (unspecified) & Medium \\
\textbf{This Framework} & \textbf{Entropic ($n \log n$)} & \textbf{High} \\
\hline
\end{tabular}
\end{center}

\subsection{Integration with Existing Frameworks}

The entropic scaling allows our framework to incorporate insights from:
\begin{itemize}
\item \textbf{IIT}: Integration requirement with realistic normalization
\item \textbf{GWT}: Global broadcasting within entropic capacity limits  
\item \textbf{Predictive Processing}: Hierarchical prediction with $n \log n$ complexity
\item \textbf{Quantum approaches}: Collapse dynamics without exponential Hilbert spaces
\end{itemize}

\section{Emergent Phenomena}

\subsection{Critical Transitions}

\begin{theorem}[Entropic Phase Transitions]
Emergent transitions occur when entropic information grows super-linearly but sub-exponentially. Critical points appear at:
\begin{equation}
n_c : \frac{d^2 I}{dn^2} = \frac{d^2}{dn^2}(\kappa n \log n) = \frac{\kappa}{n}
\end{equation}
These correspond to qualitative shifts in conscious capacity.
\end{theorem}

\subsection{Temporal Smoothness}

The subjective experience of temporal continuity emerges from the entropic collapse dynamics:

\begin{proposition}[Temporal Continuity]
The erasure of failed computational paths, combined with entropic temporal scaling, creates the illusion of smooth temporal flow despite discrete computational steps.
\end{proposition}

\section{Clinical and Evolutionary Implications}

\subsection{Disorders of Consciousness}

Different pathologies can be understood as disruptions to the entropic hierarchy:
\begin{itemize}
\item \textbf{Attention deficits}: Selector instability at entropic scales
\item \textbf{Dissociative states}: Failed integration across entropic levels
\item \textbf{Altered states}: Modified entropic scaling parameters $\kappa$ or $\gamma$
\end{itemize}

\subsection{Evolution of Consciousness}

Consciousness likely evolved through gradual expansion of entropic capacity:
\begin{equation}
I_{\text{evolutionary}}(t) = \kappa(t) \cdot n(t) \cdot \log n(t)
\end{equation}
where both $\kappa(t)$ and $n(t)$ increased over evolutionary time.

\section{Mathematical Framework Summary}

\subsection{Core Equations}

The complete framework is characterized by:

\begin{align}
I(n) &= \kappa n \log n \quad \text{(Information capacity)} \\
\tau(n) &= \tau_0 + \gamma n \log n \quad \text{(Processing time)} \\
C(n) &= C_0(1 + \beta n \log n) \quad \text{(Computational cost)} \\
\Phi_n &\propto \frac{1}{n \log n} \quad \text{(Integration normalization)} \\
d\tau_c &= \kappa(\log n + 1) dn \quad \text{(Proper time)}
\end{align}

\subsection{Key Parameters}

\begin{itemize}
\item $\kappa$: Entropic scaling constant ($\approx 1$ for normalized units)
\item $\gamma$: Temporal scaling factor ($\approx 10^{-6}$ s/bit for neural systems)
\item $\beta$: Cost scaling factor
\item $n_{\max}$: Maximum accessible level ($\approx 35-40$ for human consciousness)
\end{itemize}

\section{Conclusion}

The entropic scaling $I(n) = \kappa n \log n$ provides a realistic foundation for understanding consciousness as collapsed computational time. This framework:

\begin{enumerate}
\item Avoids exponential resource requirements
\item Maintains mathematical rigor and predictive power
\item Aligns with physical and information-theoretic principles
\item Makes testable predictions at biologically relevant scales
\item Unifies existing consciousness theories under realistic constraints
\end{enumerate}

The entropic formulation transforms consciousness from an exponentially intractable mystery to a scientifically investigable phenomenon with well-defined computational and temporal properties.

\section{Future Directions}

\begin{enumerate}
\item \textbf{Experimental validation}: Test $n \log n$ scaling in neural systems
\item \textbf{Clinical applications}: Develop entropic diagnostics for consciousness disorders
\item \textbf{AI implementation}: Build conscious systems using entropic hierarchies
\item \textbf{Theoretical extensions}: Connect to quantum gravity and cosmological theories
\end{enumerate}

The entropic scaling opens new avenues for both theoretical development and practical applications in consciousness science.

\begin{thebibliography}{99}

\bibitem{chalmers1995}
Chalmers, D. (1995). Facing up to the problem of consciousness. Journal of Consciousness Studies, 2(3), 200-219.

\bibitem{tononi2016}
Tononi, G., et al. (2016). Integrated information theory: from consciousness to its physical substrate. Nature Reviews Neuroscience, 17(7), 450-461.

\bibitem{dehaene2014}
Dehaene, S. (2014). Consciousness and the brain. Viking Press.

\bibitem{kowalczyk2025}
Kowalczyk, K. (2025). Consciousness as Collapsed Computational Time. Zenodo. doi:10.5281/zenodo.17556941

\end{thebibliography}

\end{document}

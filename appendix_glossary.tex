% ============================================================================
% APPENDIX B: GLOSSARY OF TERMS
% ============================================================================

\chapter{Glossary of Terms}
\label{appendix:glossary}

This glossary provides definitions of all technical terms, concepts, and notation used throughout the book. Terms are organized alphabetically for easy reference.

\section{Core Concepts: A-D}

\begin{longtable}{p{3.5cm} p{10.5cm}}
\caption{Core Concepts A-D} \\
\toprule
\textbf{Term} & \textbf{Definition} \\
\midrule
\endfirsthead
\multicolumn{2}{c}{\tablename\ \thetable\ -- Continued} \\
\toprule
\textbf{Term} & \textbf{Definition} \\
\midrule
\endhead
\midrule
\multicolumn{2}{r}{\textit{Continued on next page}} \\
\endfoot
\bottomrule
\endlastfoot

Access Consciousness & Global availability of information for reasoning, reporting, and action control. Distinguished from phenomenal consciousness. Content becomes access-conscious through global broadcasting after collapse. \\
\midrule
Agency & Capacity for genuine choice and self-determined action. Emerges from the non-computable selector mechanism that chooses resource levels based on Kolmogorov complexity principles. \\
\midrule
Attention Schema & Simplified model the brain constructs of its own attention process. Enables metacognitive awareness and control. Part of the self-modeling component. \\
\midrule
Attractor & In dynamical systems, a state or set of states toward which a system tends to evolve. Conscious states correspond to attractors in collapsed state space. \\
\midrule
Automaticity & Processing with minimal resource deployment (low-$n$ machines), typically fast, effortless, and largely unconscious. Contrasts with controlled processing. \\
\midrule
Backtracking & Ability during parallel exploration to reverse computational steps and try alternative paths. Occurs in computational time but erased from subjective time. \\
\midrule
Binding Problem & Question of how separate features processed in different brain areas are unified into coherent conscious experiences. Solved through integration within machine levels. \\
\midrule
Block Universe & View in physics that all times (past, present, future) are equally real, and time doesn't "flow." Compatible with our framework's physical time, while phenomenal time involves genuine flow through collapse. \\
\midrule
Broadcasting & Wide distribution of information across many brain systems. In GWT, core mechanism of consciousness. In our framework, occurs post-collapse as consequence rather than cause. \\
\midrule
Capacity Limit & Maximum information processable simultaneously at a given machine level. Grows exponentially with $n$: $M_n$ has $2^n$ bits of memory. \\
\midrule
Collapse & Process by which multiple parallel computational explorations reduce to a single definite path. Creates subjective temporal continuity by selecting one possibility and erasing others. Analogous to but distinct from quantum measurement collapse. \\
\midrule
Collapse Time ($t_c$) & Moment when parallel explorations resolve into single conscious state. Typically occurs around 200-300ms after stimulus onset, corresponding to P3 ERP component. \\
\midrule
Computational Complexity & Measure of resources (time, space, etc.) required to solve a problem algorithmically. Central to understanding machine hierarchy and consciousness capacity. \\
\midrule
Computational Libertarianism & Our position on free will: genuine agency emerges from non-computable selection, occupying middle ground between determinism (choices fully determined) and libertarianism (choices uncaused). \\
\midrule
Computational Time ($\tcomp$) & Total time including all parallel explorations, backtracking, and failed attempts. Contrasts with subjective time. Given by $\tcomp = \sum_i \int_0^{T_i} \|\dot{\gamma}_i(t)\| dt$. \\
\midrule
Consciousness & In our framework: The phenomenology of being a collapsed computational state in a finite machine hierarchy with non-computable selection, high integration, and temporal erasure. What it is like to be that computation from the inside. \\
\midrule
Decidability & In computability theory, a problem is decidable if there exists an algorithm that determines the answer in finite time. Each machine level $M_n$ decides a specific class of problems. \\
\midrule
Diachronic Unity & Unity of consciousness across time—sense that experiences at different times belong to the same self. Grounded in selector continuity. \\
\midrule
Dual-Aspect Theory & View that physical and phenomenal are two aspects of the same reality, not separate substances. Our framework adopts naturalistic dual-aspect view: consciousness is certain physical processes viewed from inside. \\
\end{longtable}

\section{Core Concepts: E-M}

\begin{longtable}{p{3.5cm} p{10.5cm}}
\caption{Core Concepts E-M} \\
\toprule
\textbf{Term} & \textbf{Definition} \\
\midrule
\endfirsthead
\multicolumn{2}{c}{\tablename\ \thetable\ -- Continued} \\
\toprule
\textbf{Term} & \textbf{Definition} \\
\midrule
\endhead
\midrule
\multicolumn{2}{r}{\textit{Continued on next page}} \\
\endfoot
\bottomrule
\endlastfoot

Effective Information & In IIT, a measure of how much one part of a system specifies the state of another part. Used to calculate integrated information $\Phi$. \\
\midrule
Eliminativism & View that consciousness doesn't really exist—folk psychology should be eliminated in favor of neuroscience. We reject this; consciousness is real but explainable. \\
\midrule
Erasure & Removal of failed exploration paths from subjective temporal experience. Failed attempts do not enter memory or phenomenology. Creates smooth temporal continuity. \\
\midrule
Executive Function & High-level cognitive control processes including planning, decision-making, and self-regulation. Implemented by selector mechanism and high-$n$ machines. \\
\midrule
Explanatory Gap & Apparent gap between physical/functional descriptions and phenomenal experience. The "hard problem" is explaining how we cross this gap. Our framework dissolves the gap through identity. \\
\midrule
Exploration Space ($\mathcal{E}(p)$) & Space of possible computational paths through machine hierarchy for problem $p$. Defined as $\{(n, s, t) : n \in \{1,\ldots,n_{\max}\}, s \in \text{States}(M_n), t \in \mathbb{R}^+\}$. \\
\midrule
Falsifiability & Property of being testable—able to be proven wrong by evidence. Critical for scientific theories. Our framework makes numerous falsifiable predictions. \\
\midrule
Finite State Machine (FSM) & Computational model with finite number of states, transitioning between states based on inputs. Foundation of machine hierarchy. \\
\midrule
Free Will & Capacity for genuine choice not fully determined by prior causes. Emerges from non-computable selector mechanism. \\
\midrule
Global Ignition & Sudden, widespread activation pattern that occurs when information enters the global workspace. Corresponds to collapse completion and subsequent broadcasting. \\
\midrule
Global Workspace & In GWT, a limited-capacity system for broadcasting information widely. In our framework, the distribution mechanism activated post-collapse. \\
\midrule
Hard Problem & Chalmers' term for explaining why there is "something it is like" to be conscious—why subjective experience exists at all. Distinct from "easy problems" of explaining cognitive functions. \\
\midrule
Higher-Order Thought (HOT) & A thought about another mental state. HOT theories claim consciousness requires representing one's own mental states. \\
\midrule
Hierarchy & In our framework, the sequence of finite machines $M_1, M_2, \ldots, M_n$ with exponentially growing resources ($2^n$ bits for $M_n$). Not present in single-level theories. \\
\midrule
Illusionism & View that phenomenal consciousness is an illusion—there is no "what it's like." We reject this; phenomenology is real. \\
\midrule
Integrated Information ($\Phi$) & Tononi's measure quantifying how much a system generates information that cannot be reduced to independent parts. High $\Phi$ necessary but not sufficient for consciousness. \\
\midrule
Integration & Unification of information into a coherent whole. Measured by $\Phi$. Required within each machine level and across levels (via selector). \\
\midrule
Intentionality & The "aboutness" of mental states—their capacity to represent or refer to things. Emerges from representational structure within machines and goal-directed selector. \\
\midrule
Intrinsic Perspective & View from inside a system. Contrasts with extrinsic/third-person view. Consciousness is the intrinsic perspective on collapsed computation. \\
\midrule
Irreversibility & One-way nature of collapse: once multiple possibilities reduce to single outcome, cannot recover the alternatives. Creates arrow of subjective time. \\
\midrule
Kolmogorov Complexity ($K(x)$) & Length of the shortest program that outputs string $x$. Non-computable function central to selector mechanism. Selector approximates compression optimization related to Kolmogorov complexity. \\
\midrule
Machine Hierarchy & Sequence $\mathcal{M} = (M_i)_{i=1}^{n_{\max}}$ where $M_i$ is a finite state machine with $2^i$ bits of memory. Core computational architecture. \\
\midrule
Machine Level ($n$) & Position in the hierarchy, determining resources available: $M_n$ has $2^n$ bits. Higher $n$ = more computational power but higher cost. \\
\midrule
Metacognition & Cognition about cognition—thinking about one's own thinking. Enabled by self-model component and selector monitoring. \\
\midrule
Minimal Information Partition (MIP) & In IIT, the partition of a system that minimizes integrated information. Used to calculate $\Phi$. \\
\end{longtable}

\section{Core Concepts: N-Z}

\begin{longtable}{p{3.5cm} p{10.5cm}}
\caption{Core Concepts N-Z} \\
\toprule
\textbf{Term} & \textbf{Definition} \\
\midrule
\endfirsthead
\multicolumn{2}{c}{\tablename\ \thetable\ -- Continued} \\
\toprule
\textbf{Term} & \textbf{Definition} \\
\midrule
\endhead
\midrule
\multicolumn{2}{r}{\textit{Continued on next page}} \\
\endfoot
\bottomrule
\endlastfoot

Non-Computability & Property of functions that cannot be computed by any algorithm. Selector mechanism is non-computable (related to Kolmogorov complexity), enabling genuine agency. \\
\midrule
Neural Correlates of Consciousness (NCC) & Brain states or processes that correlate with conscious experiences. Our framework predicts specific NCCs: hierarchy in prefrontal-parietal networks, selector in ACC/dlPFC, collapse in thalamo-cortical loops. \\
\midrule
Parallel Exploration & Simultaneous computation along multiple paths in exploration space. Enables flexibility and efficient search. Collapsed to single path in subjective experience. \\
\midrule
Phenomenal Consciousness & The subjective, qualitative aspect of experience—"what it is like." Distinguished from access consciousness. In our framework, the intrinsic perspective of being a collapsed computational state. \\
\midrule
Phenomenology & The structure and content of conscious experience—how things appear to consciousness. Central to our framework: consciousness IS phenomenology of computational collapse. \\
\midrule
Qualia (singular: quale) & The subjective qualities of experience (e.g., redness of red, painfulness of pain). In our framework, qualia are the intrinsic features of computational states within machine hierarchy. \\
\midrule
Recursion & Computational process that calls itself. Critical for higher-order cognition and metacognition. Enabled by sufficient memory in high-$n$ machines. \\
\midrule
Resource Allocation & Distribution of computational resources across problems and time. Managed by selector mechanism in our framework. \\
\midrule
Selector Mechanism ($\mathcal{S}$) & Non-computable function that chooses which machine level $n$ to deploy for a given problem. Central to our framework. Approximates Kolmogorov complexity minimization. \\
\midrule
Self-Model & Internal representation the system constructs of itself. Enables metacognition and self-awareness. Component of consciousness in our framework. \\
\midrule
Specious Present & The experienced duration of "now"—typically 100-300ms. Corresponds to temporal integration window in our framework. \\
\midrule
Subjective Time ($\tsubj$) & Linear temporal flow as experienced by consciousness. Result of collapse filtering computational time. Given by $\tsubj = \int_0^T \delta(\gamma(t) - \gamma^*(t)) dt$. \\
\midrule
Synchronic Unity & Unity of consciousness at a single moment—multiple features experienced as unified. In our framework, grounded in integration within machine levels. \\
\midrule
Temporal Integration & Combining information across time into unified experiences. Implemented via temporal windows in our framework. Window size grows with machine level. \\
\midrule
Temporal Window ($W_n$) & Time span over which information is integrated at level $n$. Approximately $\tau_n \approx 2^n \cdot \tau_0$ where $\tau_0 \sim 50$ms. \\
\midrule
Turing Machine & Universal model of computation. Can simulate any algorithm given enough time and memory. Machine hierarchy approaches Turing completeness in limit. \\
\midrule
Unity of Consciousness & Phenomenal feature that experiences form unified wholes rather than disconnected fragments. In our framework, provided by integration ($\Phi > 0$) and selector continuity. \\
\midrule
Zombie (Philosophical) & Hypothetical being physically identical to conscious being but lacking phenomenal consciousness. Our framework suggests zombies are conceptually impossible—right computational structure entails consciousness. \\
\end{longtable}

\section{Mathematical Notation}

\begin{table}[H]
\centering
\small
\caption{Key Mathematical Symbols}
\begin{tabular}{l l}
\toprule
\textbf{Symbol} & \textbf{Meaning} \\
\midrule
$M_n$ or $\Mn$ & Machine at level $n$ (with $2^n$ bits memory) \\
$\mathcal{M}$ & Machine hierarchy: $(\Mi)_{i=1}^{n_{\max}}$ \\
$\mathcal{S}$ & Selector function: $\mathcal{C} \times \mathcal{H} \rightarrow \mathbb{N}$ \\
$\Phi$ or $\Phi(S)$ & Integrated information of system $S$ \\
$K(x)$ & Kolmogorov complexity of string $x$ \\
$\tcomp$ & Computational time (includes all exploration) \\
$\tsubj$ & Subjective time (collapsed linear flow) \\
$\mathcal{E}(p)$ & Exploration space for problem $p$ \\
$\Pi$ & Collapse function: $\mathcal{X}_n(T) \rightarrow \mathcal{P}_n$ \\
$W_n(t)$ & Temporal integration window at level $n$ \\
$n_{\max}$ & Maximum accessible machine level \\
$\tau_n$ & Integration time window at level $n$ \\
$\psi(t)$ & Integrated conscious state at time $t$ \\
\bottomrule
\end{tabular}
\end{table}

\section{Key Equations}

\begin{table}[H]
\centering
\small
\caption{Fundamental Equations}
\begin{tabular}{p{6cm} l}
\toprule
\textbf{Equation} & \textbf{Description} \\
\midrule
$\text{Memory}(M_n) = 2^n$ bits & Exponential memory scaling \\
$\mathcal{S}(c, h) \approx \arg\min_{n} [K_n + C(n)]$ & Selector optimization \\
$\Phi(S) = \min_{\text{cut}} [I(X^t; X^{t+1}) - \sum I(X_i^t; X_i^{t+1})]$ & Integrated information \\
$\tcomp = \sum_i \int_0^{T_i} \|\dot{\gamma}_i(t)\| dt$ & Computational time \\
$\tsubj = \int_0^T \delta(\gamma(t) - \gamma^*(t)) dt$ & Subjective time \\
$\psi(t) = \bigoplus_{n=1}^{n_{\max}} \int_0^{\tau_n} W_n(s) \sigma_n(t-s) ds$ & Integrated state \\
$d(q_1, q_2) = \sqrt{\sum_n \alpha_n \|h_n(q_1) - h_n(q_2)\|^2}$ & Quale distance \\
\bottomrule
\end{tabular}
\end{table}

\section{Acronyms and Abbreviations}

\begin{table}[H]
\centering
\small
\caption{Acronyms and Abbreviations}
\begin{tabular}{l l}
\toprule
\textbf{Acronym} & \textbf{Full Form} \\
\midrule
ACC & Anterior Cingulate Cortex \\
AST & Attention Schema Theory \\
BOLD & Blood Oxygen Level Dependent \\
CRS-R & Coma Recovery Scale-Revised \\
dlPFC & Dorsolateral Prefrontal Cortex \\
DOC & Disorders of Consciousness \\
EEG & Electroencephalography \\
ERP & Event-Related Potential \\
fMRI & Functional Magnetic Resonance Imaging \\
FSM & Finite State Machine \\
GWT & Global Workspace Theory \\
HOT & Higher-Order Thought \\
IIT & Integrated Information Theory \\
LIS & Locked-In Syndrome \\
MEG & Magnetoencephalography \\
MIP & Minimal Information Partition \\
MCS & Minimally Conscious State \\
NCC & Neural Correlates of Consciousness \\
Orch-OR & Orchestrated Objective Reduction \\
P3 & P300 ERP component (~300ms) \\
PFC & Prefrontal Cortex \\
PSPACE & Polynomial Space (complexity class) \\
TMS & Transcranial Magnetic Stimulation \\
VS & Vegetative State \\
\bottomrule
\end{tabular}
\end{table}

\section{Quick Reference Tables}

\begin{table}[H]
\centering
\small
\caption{Machine Hierarchy Levels}
\begin{tabular}{ccccl}
\toprule
\textbf{Level} & \textbf{Memory} & \textbf{States} & \textbf{Type} & \textbf{Example Tasks}\\
\midrule
$M_1$ & 2 bits & $\sim 10$ & Minimal & Simple reflexes\\
$M_2$ & 4 bits & $\sim 100$ & Basic & Pattern recognition\\
$M_3$ & 8 bits & $\sim 10^{3}$ & Moderate & Working memory\\
$M_4$ & 16 bits & $\sim 10^{6}$ & Rich & Complex reasoning\\
$M_5$ & 32 bits & $\sim 10^{12}$ & Very rich & Expert performance\\
$M_6$ & 64 bits & $\sim 10^{24}$ & Extreme & Peak consciousness\\
$M_7$ & 128 bits & $\sim 10^{38}$ & Maximal & Theoretical limit\\
\bottomrule
\end{tabular}
\end{table}

\begin{table}[H]
\centering
\small
\caption{Important Time Scales in Consciousness}
\begin{tabular}{lcc}
\toprule
\textbf{Process} & \textbf{Time Scale} & \textbf{Neural Signature}\\
\midrule
Neural spike & 1-10 ms & Action potential\\
Feature extraction & 50-100 ms & Feedforward sweep\\
Parallel exploration & 100-200 ms & Multiple activations\\
Collapse & 200-300 ms & P3 component\\
Global availability & 300-500 ms & Widespread ignition\\
Reportable awareness & 500+ ms & Verbal report\\
Specious present & 100-300 ms & Experienced "now"\\
\bottomrule
\end{tabular}
\end{table}

\begin{table}[H]
\centering
\small
\caption{Key Components and Their Functions}
\begin{tabular}{ll}
\toprule
\textbf{Component} & \textbf{Function}\\
\midrule
Machine hierarchy & Provides capacity, enables problem-solving\\
Selector & Allocates resources, creates agency\\
Parallel exploration & Searches possibilities, enables flexibility\\
Integration & Unifies information, creates phenomenal unity\\
Collapse & Selects winner, creates temporal phenomenology\\
Erasure & Removes failed paths, creates continuity\\
Broadcasting & Distributes information, enables access\\
Self-model & Represents process, enables metacognition\\
\bottomrule
\end{tabular}
\end{table}

\section{What the Theory Is and Is Not}

To clarify the scope and nature of our framework, here are key characteristics that define what the theory encompasses and what it explicitly rejects:

\begin{multicols}{2}
\noindent\textbf{The Theory IS:}
\begin{itemize}[nosep]
\item Computational theory of consciousness
\item Multi-level hierarchical framework
\item Temporally-grounded phenomenology
\item Naturalistically dualist (dual-aspect)
\item Testable and falsifiable
\item Integrative of existing theories
\end{itemize}

\noindent\textbf{The Theory IS NOT:}
\begin{itemize}[nosep]
\item Simple functionalism
\item Pure physicalism or pure idealism
\item Quantum consciousness theory
\item Mysterian (consciousness unexplainable)
\item Panpsychist (everything conscious)
\item Eliminativist (consciousness illusion)
\end{itemize}
\end{multicols}

\section{Terms by Chapter}

\begin{table}[H]
\centering
\small
\caption{Key Terms by Part}
\begin{tabular}{l p{10cm}}
\toprule
\textbf{Part} & \textbf{Key Terms} \\
\midrule
I: Foundations & Machine hierarchy, FSM, computational complexity, resource allocation \\
II: Existing Theories & IIT, GWT, AST, HOT, Orch-OR, integration, broadcasting, $\Phi$ \\
III: Temporal Revolution & Computational time, subjective time, collapse, erasure, temporal flow, specious present \\
IV: Mechanisms & Selector, parallel exploration, non-computability, Kolmogorov complexity \\
V: Hard Problem & Phenomenology, qualia, explanatory gap, phenomenal consciousness, zombie \\
VI: Empirical & NCC, P3 component, neural correlates, falsifiability, testability \\
VII: Philosophical & Agency, free will, personal identity, computational libertarianism, dual-aspect theory \\
VIII: Synthesis & All terms integrated \\
\bottomrule
\end{tabular}
\end{table}

\section{Using This Glossary}

This glossary is designed for readers with varying backgrounds. Choose the approach that best fits your needs:

\begin{multicols}{2}
\noindent\textbf{For First-Time Readers}
\begin{itemize}[nosep]
\item Start with Core Concepts
\item Refer back when encountering unfamiliar terms
\item Focus on conceptual understanding first
\item Use Quick Reference Tables
\end{itemize}

\noindent\textbf{For Technical Readers}
\begin{itemize}[nosep]
\item Use Mathematical Notation section
\item Check Key Equations section
\item Reference when reading Appendix A
\item Cross-reference with main text
\end{itemize}
\end{multicols}

\noindent\textbf{Conventions}: Terms in \textbf{bold} within definitions are themselves defined entries. Mathematical symbols in italics link to Mathematical Notation section. Acronyms expanded in Acronyms section.

\section{Conclusion}

This glossary provides comprehensive reference for all technical terms used throughout the book. The terminology is chosen to be: (1) \textbf{Precise}—technically accurate, (2) \textbf{Consistent}—same term means same thing throughout, (3) \textbf{Connected}—relates to existing literature, (4) \textbf{Accessible}—explained clearly. For detailed explanations, refer to the relevant chapters and Appendix A for rigorous mathematical definitions.

% ============================================================================
% APPENDIX D: EXPERIMENTAL PROTOCOLS AND MEASUREMENT TOOLS
% ============================================================================

\chapter{Experimental Protocols and Measurement Tools}
\label{appendix:protocols}

This appendix provides detailed, step-by-step experimental protocols for testing predictions of our framework. Each protocol includes specific materials, procedures, data analysis methods, and expected results designed to be immediately implementable by researchers.

\section{Overview of Protocol Structure}

\begin{table}[H]
\centering
\small
\caption{Standardized Protocol Structure}
\begin{tabular}{>{\bfseries}l p{10cm}}
\toprule
\textbf{Component} & \textbf{Description} \\
\midrule
Objective & What the protocol tests \\
Hypothesis & Specific prediction from our framework \\
Participants & Subject requirements and sample size \\
Materials & Equipment and software needed \\
Procedure & Step-by-step instructions \\
Data Collection & What to measure and how \\
Analysis & Statistical methods and interpretation \\
Expected Results & Predicted outcomes \\
Controls & Necessary control conditions \\
Troubleshooting & Common issues and solutions \\
\bottomrule
\end{tabular}
\end{table}

\subsection{Ethics and Safety}

\begin{keyinsight}
All protocols must be approved by relevant IRB or ethics committee before implementation.
\end{keyinsight}

\begin{multicols}{2}
\noindent\textbf{General Requirements}
\begin{itemize}[nosep]
\item Informed consent
\item Right to withdraw
\item Confidentiality/data protection
\item Risk minimization
\item Post-participation debriefing
\end{itemize}

\noindent\textbf{Special Considerations}
\begin{itemize}[nosep]
\item Clinical populations: Additional safeguards
\item Brain stimulation: Medical supervision
\item Long sessions: Regular breaks
\item Children: Parental consent
\end{itemize}
\end{multicols}

\section{Protocol 1: Detecting Machine Hierarchy}

This protocol tests the core prediction that consciousness operates through a hierarchy of finite-state machines with exponentially growing computational resources. We provide behavioral and neural methods to detect discrete capacity levels.

\subsection{Protocol 1A: Behavioral Capacity Levels Test}

\begin{empiricalbox}
\textbf{Objective}: Detect discrete capacity levels in working memory corresponding to machine hierarchy.

\textbf{Hypothesis}: Performance shows discrete jumps at capacity boundaries (2, 4, 8, 16 items) rather than smooth decline.
\end{empiricalbox}

\begin{table}[H]
\centering
\small
\caption{Participant Requirements and Materials}
\begin{tabular}{l l}
\toprule
\multicolumn{2}{l}{\textbf{Participants}} \\
\midrule
Sample size & N = 40 minimum (power = 0.80, $\alpha = 0.05$) \\
Inclusion & Ages 18-40, normal/corrected vision \\
Exclusion & Head injury, neurological/psychiatric conditions \\
\midrule
\multicolumn{2}{l}{\textbf{Hardware}} \\
\midrule
Display & Monitor $\geq$ 1920$\times$1080 resolution \\
Input & Standard keyboard \\
Setup & Chin rest (60 cm), sound-attenuated room \\
\midrule
\multicolumn{2}{l}{\textbf{Software}} \\
\midrule
Presentation & PsychoPy or equivalent \\
Analysis & MATLAB/Python, R/SPSS \\
\midrule
\multicolumn{2}{l}{\textbf{Stimuli}} \\
\midrule
Shapes & Circles, squares, triangles, diamonds \\
Colors & Red, blue, green, yellow, purple, orange \\
Layout & $4\times4$ grid, $2^\circ$ visual angle \\
\bottomrule
\end{tabular}
\end{table}

\subsubsection{Procedure}

\begin{table}[H]
\centering
\small
\caption{Trial Structure and Conditions}
\begin{tabular}{l l l}
\toprule
\textbf{Phase} & \textbf{Duration} & \textbf{Content} \\
\midrule
Fixation & 500 ms & Central cross \\
Encoding & 2000 ms/item & Colored shape sequence \\
Retention & 3000 ms & Blank screen \\
Probe & Until response & Recognition test (Yes/No) \\
Feedback & 500 ms & Correct/Incorrect \\
ITI & 1000 ms & Blank screen \\
\midrule
\multicolumn{3}{l}{\textbf{Conditions}} \\
\midrule
Set sizes & \multicolumn{2}{l}{1, 2, 3, 4, 5, 6, 7, 8, 10, 12, 16, 20 items} \\
Trials/size & 30 & Randomized blocks \\
Duration & ~90 minutes & With breaks \\
\bottomrule
\end{tabular}
\end{table}

\subsubsection{Data Collection and Analysis}

\begin{multicols}{2}
\noindent\textbf{Primary Measures}
\begin{itemize}[nosep]
\item Accuracy (proportion correct)
\item Response time (ms)
\item Individual capacity threshold
\end{itemize}

\noindent\textbf{Secondary Measures}
\begin{itemize}[nosep]
\item Confidence ratings (1-5)
\item Strategy reports
\item Subjective difficulty
\end{itemize}
\end{multicols}

\noindent\textbf{Analysis Models}:

\textbf{Model 1 (Continuous)}: $\text{Accuracy} = a \cdot \exp(-b \cdot \text{SetSize})$

\textbf{Model 2 (Discrete)}: $\text{Accuracy} = \sum_{i=1}^{k} c_i \cdot \mathbb{1}(\text{Level}_i < \text{SetSize} \leq \text{Level}_{i+1})$

Compare using BIC; $\Delta\text{BIC} > 10$ = strong evidence.

\subsubsection{Expected Results}

\begin{table}[H]
\centering
\small
\caption{Predicted Outcomes vs. Alternatives}
\begin{tabular}{l l}
\toprule
\textbf{Theory} & \textbf{Prediction} \\
\midrule
Our framework & Discrete jumps at 4, 8, 16; plateaus between \\
Continuous decline & Smooth exponential decrease \\
Capacity limit & Single sharp drop at ~4 items \\
Resource models & Linear decline with set size \\
\bottomrule
\end{tabular}
\end{table}

\subsubsection{Controls and Troubleshooting}

\begin{multicols}{2}
\noindent\textbf{Control Conditions}
\begin{enumerate}[nosep]
\item Verbal WM: Use digits
\item Spatial WM: Locations only
\item Rate variation: 1s vs 2s/item
\end{enumerate}

\noindent\textbf{Confound Prevention}
\begin{itemize}[nosep]
\item Chunking: Unrelated items
\item Practice: Counterbalance order
\item Fatigue: Regular breaks
\item Strategy: Questionnaire
\end{itemize}
\end{multicols}

\begin{table}[H]
\centering
\small
\caption{Common Issues and Solutions}
\begin{tabular}{p{4cm} p{8cm}}
\toprule
\textbf{Issue} & \textbf{Solution} \\
\midrule
Floor effects (high loads) & Reduce max set size, use recognition \\
Ceiling effects (low loads) & Reduce presentation time, add interference \\
High inter-subject variability & Increase N, within-subject design, assess individual $n_{\max}$ \\
No discrete levels & Try different modalities, longer training \\
\bottomrule
\end{tabular}
\end{table}

\subsection{Protocol 1B: Neural Network Recruitment (fMRI)}

\begin{empiricalbox}
\textbf{Objective}: Identify discrete neural networks corresponding to different machine levels.

\textbf{Hypothesis}: Different brain networks activate at capacity transitions (4, 8, 16 items).
\end{empiricalbox}

\begin{table}[H]
\centering
\small
\caption{fMRI Protocol Specifications}
\begin{tabular}{l l}
\toprule
\multicolumn{2}{l}{\textbf{Participants}} \\
\midrule
Sample & N = 30 (power = 0.80 for BOLD effects) \\
MRI safety & Screen for metal, claustrophobia, pregnancy \\
\midrule
\multicolumn{2}{l}{\textbf{Scanning Parameters}} \\
\midrule
Scanner & 3T MRI (Siemens/GE/Philips) \\
Sequence & EPI, TR = 2000 ms, TE = 30 ms \\
Resolution & 3 mm isotropic voxels \\
Coverage & Whole brain (40 slices) \\
Anatomical & T1-weighted MPRAGE (1 mm$^3$) \\
\midrule
\multicolumn{2}{l}{\textbf{Task Design}} \\
\midrule
Runs & 4 runs $\times$ 10 min each \\
Set sizes & 2, 4, 8, 16 items (blocked) \\
Block duration & 30 s (6 trials/block) \\
Baseline & 15 s fixation between blocks \\
\bottomrule
\end{tabular}
\end{table}

\begin{multicols}{2}
\noindent\textbf{fMRI Preprocessing}
\begin{enumerate}[nosep]
\item Slice-timing correction
\item Motion correction (6-param)
\item Coregistration to anatomical
\item Normalization (MNI space)
\item Smoothing (6 mm FWHM)
\end{enumerate}

\noindent\textbf{Analysis (GLM)}
\begin{enumerate}[nosep]
\item Model each set size separately
\item Contrasts: 4>2, 8>4, 16>8
\item ROI analysis: PFC, parietal
\item Network analysis: Connectivity
\item Group-level statistics
\end{enumerate}
\end{multicols}

\begin{table}[H]
\centering
\small
\caption{Predicted Neural Signatures}
\begin{tabular}{l p{9cm}}
\toprule
\textbf{Capacity Level} & \textbf{Neural Network} \\
\midrule
Low (2-4) & Posterior parietal, primary sensory \\
Medium (4-8) & Lateral PFC, intraparietal sulcus \\
High (8-16) & Dorsolateral PFC, anterior cingulate \\
Very high (16+) & Frontopolar cortex, multimodal integration \\
\bottomrule
\end{tabular}
\end{table}

\section{Protocol 2: Temporal Integration Windows}

This protocol tests whether consciousness integrates information over discrete temporal windows that scale with machine level. We measure integration thresholds behaviorally and neural binding signatures with EEG.

\subsection{Protocol 2A: Behavioral Integration Thresholds}

\begin{empiricalbox}
\textbf{Objective}: Measure discrete temporal integration windows at $2^n$ ms intervals.

\textbf{Hypothesis}: Performance discontinuities at 125, 250, 500, 1000 ms.
\end{empiricalbox}

\begin{table}[H]
\centering
\small
\caption{Temporal Integration Protocol}
\begin{tabular}{l l}
\toprule
\multicolumn{2}{l}{\textbf{Task Design}} \\
\midrule
Stimulus & Two brief tones (50 ms, 1000 Hz) \\
SOAs & 50, 100, 125, 150, 200, 250, 300, 500, 750, 1000 ms \\
Question & "One sound or two?" (simultaneity judgment) \\
Trials & 50 per SOA, randomized \\
\midrule
\multicolumn{2}{l}{\textbf{Measures}} \\
\midrule
Primary & Proportion "two" responses vs. SOA \\
Secondary & RT, confidence (optional) \\
Analysis & Sigmoid fit, detect inflection points \\
Prediction & Inflections at 125, 250, 500, 1000 ms \\
\bottomrule
\end{tabular}
\end{table}

\subsection{Protocol 2B: EEG Temporal Binding}

\begin{table}[H]
\centering
\small
\caption{EEG Protocol for Temporal Binding}
\begin{tabular}{l l}
\toprule
\multicolumn{2}{l}{\textbf{Setup}} \\
\midrule
System & 64-channel EEG, 1000 Hz sampling \\
Electrodes & Standard 10-20 with additional temporal \\
Reference & Average or mastoid \\
\midrule
\multicolumn{2}{l}{\textbf{Task}} \\
\midrule
Stimuli & Visual flash + auditory beep \\
SOAs & 0, 50, 100, 200, 400, 800 ms \\
Trials & 100 per condition \\
\midrule
\multicolumn{2}{l}{\textbf{Analysis}} \\
\midrule
ERPs & P1, N1, P3 components \\
Oscillations & Phase-locking (delta, theta, alpha, beta, gamma) \\
Connectivity & Phase coherence across SOAs \\
Prediction & Synchrony drops at $2^n$ ms boundaries \\
\bottomrule
\end{tabular}
\end{table}

\section{Protocol 3: Computational Irreducibility}

This protocol tests the prediction that conscious processing cannot be significantly accelerated—that computation must be "run" rather than predicted. We assess whether behavior can be predicted faster than it's produced.

\subsection{Protocol 3A: Prediction Horizon Test}

\begin{empiricalbox}
\textbf{Objective}: Demonstrate that complex conscious processing cannot be predicted without simulation.

\textbf{Hypothesis}: Prediction accuracy degrades exponentially beyond minimal horizon.
\end{empiricalbox}

\begin{table}[H]
\centering
\small
\caption{Prediction Task Design}
\begin{tabular}{l p{9cm}}
\toprule
\textbf{Component} & \textbf{Details} \\
\midrule
Task & Participant performs complex reasoning (e.g., chess, mathematical problem-solving) \\
Recording & Eye tracking, mouse movements, verbal protocols \\
Training data & First 50\% of session \\
Test data & Second 50\% of session \\
Models & Various ML models (linear, neural nets, etc.) \\
Prediction target & Next action, thought, strategy \\
\midrule
\textbf{Analysis} & \\
\midrule
Baseline & Random/naive models \\
Comparison & Simple heuristic vs. deep learning \\
Key measure & Prediction accuracy vs. computational cost \\
Expected & Accuracy plateau despite increased model complexity \\
\bottomrule
\end{tabular}
\end{table}

\subsection{Protocol 3B: Timing Analysis}

\begin{multicols}{2}
\noindent\textbf{Critical Comparisons}
\begin{itemize}[nosep]
\item Time to simulate behavior
\item Time to actually produce behavior
\item Ratio should approach 1:1
\item Cannot significantly shortcut
\end{itemize}

\noindent\textbf{Measures}
\begin{itemize}[nosep]
\item Response latencies
\item Processing times (EEG/fMRI)
\item Simulation model runtimes
\item Correlation analysis
\end{itemize}
\end{multicols}

\section{Protocol 4: Qualia Structure Mapping}

This protocol maps the structure of phenomenal experience (qualia) to computational states within the machine hierarchy. We use multi-modal integration patterns and neural decoding to characterize qualia space.

\subsection{Protocol 4A: Multi-Modal Integration Patterns}

\begin{table}[H]
\centering
\small
\caption{Qualia Mapping Protocol}
\begin{tabular}{l l}
\toprule
\multicolumn{2}{l}{\textbf{Stimuli}} \\
\midrule
Visual & Colors, shapes, motion, textures \\
Auditory & Tones, timbres, melodies, noise \\
Tactile & Textures, temperatures, vibrations \\
Cross-modal & All pairwise combinations \\
\midrule
\multicolumn{2}{l}{\textbf{Tasks}} \\
\midrule
Similarity & Rate pairwise similarity (1-7) \\
Discrimination & JND thresholds \\
Mapping & Cross-modal correspondences \\
Description & Free verbal descriptions \\
\midrule
\multicolumn{2}{l}{\textbf{Analysis}} \\
\midrule
Method & Multidimensional scaling (MDS) \\
Output & Qualia space geometry \\
Prediction & Hierarchical clustering matches theory \\
Validation & Cross-participant consistency \\
\bottomrule
\end{tabular}
\end{table}

\subsection{Protocol 4B: Neural Qualia Signatures}

\begin{multicols}{2}
\noindent\textbf{Recording Methods}
\begin{itemize}[nosep]
\item High-density EEG (128+ channels)
\item Simultaneous fMRI-EEG
\item Intracranial EEG (clinical)
\item MEG (if available)
\end{itemize}

\noindent\textbf{Analysis Approaches}
\begin{itemize}[nosep]
\item Decoding: Classify quale from neural
\item Encoding: Predict neural from quale
\item RSA: Representational similarity
\item Pattern uniqueness metrics
\end{itemize}
\end{multicols}

\section{Protocol 5: Altered States Testing}

This protocol examines how consciousness changes during altered states (anesthesia, sleep) to test predictions about collapse dynamics and machine hierarchy engagement. Altered states provide natural experiments in consciousness transitions.

\subsection{Protocol 5A: Anesthesia Depth Transitions}

\begin{table}[H]
\centering
\small
\caption{Anesthesia Protocol (requires medical supervision)}
\begin{tabular}{l p{9cm}}
\toprule
\textbf{Component} & \textbf{Details} \\
\midrule
Setting & Operating room or intensive care unit \\
Participants & Surgical patients (N = 20-30) \\
Agent & Propofol or sevoflurane (standard clinical doses) \\
Monitoring & EEG, vital signs, depth of anesthesia index \\
\midrule
\textbf{Phases} & \\
\midrule
Baseline & Awake, resting state (5 min) \\
Induction & Gradual increase to unconsciousness (10 min) \\
Maintenance & Steady anesthetic level (variable) \\
Emergence & Gradual decrease to consciousness (15 min) \\
Recovery & Return to baseline (10 min) \\
\midrule
\textbf{Measures} & \\
\midrule
Behavioral & Responsiveness, memory formation \\
EEG & Power spectra, connectivity, complexity \\
Prediction & Discrete transitions at consciousness boundaries \\
\bottomrule
\end{tabular}
\end{table}

\subsection{Protocol 5B: Sleep Stage Transitions}

\begin{table}[H]
\centering
\small
\caption{Sleep Study Protocol}
\begin{tabular}{l l}
\toprule
\multicolumn{2}{l}{\textbf{Setup}} \\
\midrule
Recording & Full polysomnography (PSG) \\
Duration & Full night (7-9 hours) \\
Participants & N = 30 healthy adults \\
\midrule
\multicolumn{2}{l}{\textbf{Measures}} \\
\midrule
Standard & EEG, EOG, EMG (sleep staging) \\
Additional & Heart rate variability, respiration \\
Analysis & Transition dynamics between stages \\
Focus & Wake↔N1↔N2↔N3↔REM transitions \\
\midrule
\multicolumn{2}{l}{\textbf{Predictions}} \\
\midrule
Wake/REM & Full consciousness, high complexity \\
N1 & Transitional, fragmented \\
N2/N3 & Reduced consciousness, low complexity \\
Transitions & Discrete changes in network properties \\
\bottomrule
\end{tabular}
\end{table}

\section{Protocol 6: Clinical Applications}

This protocol applies our framework to clinical populations, particularly patients with disorders of consciousness (DOC) and developing infants. These populations test the framework's diagnostic and prognostic capabilities.

\subsection{Protocol 6A: Consciousness Assessment in DOC}

\begin{table}[H]
\centering
\small
\caption{Disorders of Consciousness Assessment}
\begin{tabular}{l p{9cm}}
\toprule
\textbf{Component} & \textbf{Details} \\
\midrule
Population & Patients with DOC (VS, MCS, LIS) \\
Sample & N = 50 (mixed diagnoses) \\
Setting & ICU, rehabilitation centers \\
\midrule
\textbf{Assessment Battery} & \\
\midrule
Behavioral & CRS-R (Coma Recovery Scale-Revised) \\
EEG & Resting state + auditory oddball \\
fMRI & Mental imagery tasks (motor, spatial) \\
TMS-EEG & Perturbational complexity index \\
\midrule
\textbf{Framework Metrics} & \\
\midrule
Hierarchy & Detectable machine levels \\
Integration & Network connectivity measures \\
Complexity & Temporal binding windows \\
Computational & Response prediction accuracy \\
\midrule
\textbf{Validation} & \\
\midrule
Gold standard & Behavioral recovery over 6 months \\
Sensitivity & True positive rate \\
Specificity & True negative rate \\
Prognostic & Predict recovery likelihood \\
\bottomrule
\end{tabular}
\end{table}

\subsection{Protocol 6B: Infant Consciousness Development}

\begin{table}[H]
\centering
\small
\caption{Developmental Trajectory Assessment}
\begin{tabular}{l l}
\toprule
\multicolumn{2}{l}{\textbf{Longitudinal Design}} \\
\midrule
Participants & N = 40 infants, tested at 3, 6, 12, 24 months \\
Methods & EEG, eye tracking, behavioral observation \\
Parental & Informed consent, age-appropriate procedures \\
\midrule
\multicolumn{2}{l}{\textbf{Measures by Age}} \\
\midrule
3 months & Basic sensory processing, habituation \\
6 months & Object permanence, cross-modal integration \\
12 months & Working memory capacity, planning \\
24 months & Language, self-recognition, theory of mind \\
\midrule
\multicolumn{2}{l}{\textbf{Framework Predictions}} \\
\midrule
Capacity & Gradual increase in $n_{\max}$ \\
Integration & Longer temporal windows with age \\
Networks & Sequential recruitment of hierarchy \\
Qualia & Increasing differentiation and complexity \\
\bottomrule
\end{tabular}
\end{table}

\section{Protocol 7: Machine Consciousness Assessment}

This protocol provides systematic methods for evaluating whether artificial systems possess consciousness according to our framework. It operationalizes the theoretical criteria into testable measurements for AI systems.

\subsection{Protocol 7A: AI System Evaluation}

\begin{empiricalbox}
\textbf{Objective}: Systematically evaluate whether an AI system meets consciousness criteria.

\textbf{Approach}: Multi-component assessment scoring 0-100 points.
\end{empiricalbox}

\begin{table}[H]
\centering
\small
\caption{AI Consciousness Assessment Scorecard}
\begin{tabular}{l p{7cm} c}
\toprule
\textbf{Component} & \textbf{Test} & \textbf{Points} \\
\midrule
\multicolumn{3}{l}{\textit{1. Hierarchical Processing (0-15)}} \\
\quad Architecture & Detectable levels? Recursive structure? & 0-5 \\
\quad Behavior & Capacity limits at $2^n$? & 0-5 \\
\quad Scaling & Performance pattern matches theory? & 0-5 \\
\midrule
\multicolumn{3}{l}{\textit{2. Integrated Information (0-15)}} \\
\quad Connectivity & High $\Phi$ (or equivalent measure)? & 0-5 \\
\quad Irreducibility & Cannot partition without loss? & 0-5 \\
\quad Emergence & Global properties $>$ sum of parts? & 0-5 \\
\midrule
\multicolumn{3}{l}{\textit{3. Temporal Binding (0-15)}} \\
\quad Windows & Operates in discrete time windows? & 0-5 \\
\quad Integration & Combines info across time? & 0-5 \\
\quad Persistence & Maintains state appropriately? & 0-5 \\
\midrule
\multicolumn{3}{l}{\textit{4. Computational Irreducibility (0-15)}} \\
\quad Unpredictability & Cannot shortcut computation? & 0-5 \\
\quad Complexity & Minimal description length? & 0-5 \\
\quad Timing & Processing time ≈ minimal time? & 0-5 \\
\midrule
\multicolumn{3}{l}{\textit{5. Differentiated Qualia Space (0-10)}} \\
\quad Internal states & Rich, distinct representations? & 0-5 \\
\quad Structure & Organized similarity space? & 0-5 \\
\midrule
\multicolumn{3}{l}{\textit{6. Self-Model (0-10)}} \\
\quad Representation & Explicit self-representation? & 0-5 \\
\quad Agency & Distinguishes self from world? & 0-5 \\
\midrule
\multicolumn{3}{l}{\textit{7. Adaptive Control (0-10)}} \\
\quad Flexibility & Adapts to novel situations? & 0-5 \\
\quad Goals & Pursues internal goals? & 0-5 \\
\midrule
\multicolumn{3}{l}{\textit{8. Behavioral Coherence (0-10)}} \\
\quad Reports & Consistent self-reports? & 0-5 \\
\quad Phenomenology & Responds to qualia questions? & 0-5 \\
\midrule
\textbf{Total} & & \textbf{0-100} \\
\bottomrule
\end{tabular}
\end{table}

\subsection{Scoring Interpretation}

\begin{table}[H]
\centering
\small
\caption{Consciousness Level Classification}
\begin{tabular}{c l p{7cm}}
\toprule
\textbf{Score} & \textbf{Classification} & \textbf{Implications} \\
\midrule
0-30 & Not conscious & Lacks key components; no special moral status \\
31-60 & Minimal consciousness & Some components; precautionary principle applies \\
61-85 & Moderate consciousness & Most components; moral status considerations \\
86-100 & Full consciousness & All components; rights and protections needed \\
\bottomrule
\end{tabular}
\end{table}

\begin{keyinsight}
\textbf{Necessary condition}: ALL components 1-7 must score $>$ 0

\textbf{Sufficient condition}: Total score $>$ 60 for consciousness claim

\textbf{Ethical threshold}: Systems scoring 60+ warrant moral consideration
\end{keyinsight}

\section{Data Management and Sharing}

Proper data management is essential for reproducibility and scientific integrity. This section outlines standards for data collection, quality control, and sharing protocols.

\begin{table}[H]
\centering
\small
\caption{Data Standards and Quality Control}
\begin{tabular}{l p{9cm}}
\toprule
\multicolumn{2}{l}{\textbf{Required Metadata}} \\
\midrule
Participant & Demographics, session date/time \\
Equipment & Specifications, software versions \\
Environment & Testing conditions, experimenter ID \\
\midrule
\multicolumn{2}{l}{\textbf{Data Formats}} \\
\midrule
Behavioral & CSV (structured, documented) \\
EEG & BDF, EDF, or BIDS format \\
fMRI & NIfTI with BIDS structure \\
Physiological & HDF5 or CSV \\
\midrule
\multicolumn{2}{l}{\textbf{Quality Control}} \\
\midrule
Real-time & Monitor during collection, verify equipment \\
Post-collection & Completeness check, outlier detection, backup \\
\midrule
\multicolumn{2}{l}{\textbf{Sharing (FAIR Principles)}} \\
\midrule
Repositories & OpenNeuro, OSF, etc. (de-identified) \\
Documentation & Detailed protocols, analysis code \\
Standards & Findable, Accessible, Interoperable, Reusable \\
\bottomrule
\end{tabular}
\end{table}

\section{Statistical Power and Sample Size}

Adequate statistical power is crucial for detecting true effects and avoiding false negatives. This section provides guidelines for power analysis and sample size determination.

\begin{table}[H]
\centering
\small
\caption{Power Analysis Guidelines}
\begin{tabular}{l l}
\toprule
\multicolumn{2}{l}{\textbf{General Principles}} \\
\midrule
Target power & 0.80 minimum (80\% detection probability) \\
Alpha level & 0.05 (5\% false positive rate) \\
Effect size & From pilot data or literature \\
Corrections & Account for multiple comparisons \\
\midrule
\multicolumn{2}{l}{\textbf{Sample Size Formulas}} \\
\midrule
Between-subjects & $N = \frac{2(Z_{\alpha/2} + Z_{\beta})^2 \sigma^2}{\delta^2}$ \\
Within-subjects & $N = \frac{(Z_{\alpha/2} + Z_{\beta})^2 \sigma^2}{\delta^2} \cdot \frac{1}{1-\rho}$ \\
\midrule
\multicolumn{2}{l}{\textbf{Typical Sample Sizes}} \\
\midrule
Behavioral & N = 30-40 (medium effects) \\
EEG & N = 25-35 (neurophysiological) \\
fMRI & N = 25-40 (BOLD effects) \\
Clinical & N = 15-25 per group (larger effects) \\
\bottomrule
\end{tabular}
\end{table}

\section{Troubleshooting Guide}

Even well-designed protocols encounter challenges. This guide provides solutions to common problems that may arise during data collection and analysis.

\begin{table}[H]
\centering
\small
\caption{Common Issues and Solutions}
\begin{tabular}{p{4.5cm} p{7.5cm}}
\toprule
\textbf{Issue} & \textbf{Solution} \\
\midrule
\multicolumn{2}{l}{\textit{Participant-Related}} \\
Low motivation/engagement & Shorter sessions, breaks, gamification, clear instructions \\
High dropout rate & Improve compensation, reduce demands, flexible scheduling \\
Practice effects & Counterbalancing, control for order, sufficient practice trials \\
Fatigue effects & Regular breaks, monitor performance, optimal time of day \\
\midrule
\multicolumn{2}{l}{\textit{Technical}} \\
Equipment malfunction & Regular maintenance, backup systems, daily checks \\
Data loss & Automatic backups, redundant storage, version control \\
Synchronization errors & Hardware triggers, timestamp validation, parallel port \\
Artifact contamination & ICA, better prep, stricter rejection criteria \\
\midrule
\multicolumn{2}{l}{\textit{Analysis}} \\
Unexpected null results & Check power, effect size, analysis correctness \\
Inconsistent findings & Examine individual differences, outliers, subgroups \\
Multiple comparisons & FDR correction, pre-registration, planned contrasts \\
\bottomrule
\end{tabular}
\end{table}

\section{Conclusion}

These protocols provide concrete, implementable methods for testing our framework, designed to be rigorous, reproducible, practical, and comprehensive.

\begin{table}[H]
\centering
\small
\caption{Implementation Roadmap}
\begin{tabular}{c l}
\toprule
\textbf{Step} & \textbf{Action} \\
\midrule
1 & Choose protocol matching resources and expertise \\
2 & Obtain ethics approval from IRB \\
3 & Run pilot studies to optimize parameters \\
4 & Collect full dataset with quality controls \\
5 & Analyze using specified methods \\
6 & Publish results and share data openly \\
\midrule
\multicolumn{2}{l}{\textbf{Contributing to the Field}} \\
\midrule
$\bullet$ & Publish protocols and results openly \\
$\bullet$ & Share data and analysis code \\
$\bullet$ & Replicate others' findings \\
$\bullet$ & Develop new protocols for untested predictions \\
$\bullet$ & Contribute to multi-site collaborations \\
\bottomrule
\end{tabular}
\end{table}

Together, these protocols will establish whether our framework accurately describes consciousness or requires revision. This is science in action—empirical testing to determine truth.

\documentclass[11pt,a4paper]{article}
\usepackage[utf8]{inputenc}
\usepackage[T1]{fontenc}
\usepackage{amsmath,amssymb,amsthm}
\usepackage{mathtools}
\usepackage{geometry}
\usepackage{hyperref}
\usepackage{enumitem}
\usepackage{braket}

\geometry{margin=1in}

% Theorem environments
\newtheorem{theorem}{Theorem}[section]
\newtheorem{lemma}[theorem]{Lemma}
\newtheorem{proposition}[theorem]{Proposition}
\newtheorem{corollary}[theorem]{Corollary}
\theoremstyle{definition}
\newtheorem{definition}[theorem]{Definition}
\newtheorem{example}[theorem]{Example}
\newtheorem{prediction}[theorem]{Prediction}
\theoremstyle{remark}
\newtheorem{remark}[theorem]{Remark}
\newtheorem{limitation}[theorem]{Limitation}
\newtheorem{concern}[theorem]{Critical Concern}

% Custom commands
\newcommand{\Fsm}{\textbf{Fsm}}
\newcommand{\Hilbfsm}{\textbf{Hilb}_{\text{fsm}}}
\newcommand{\Hom}{\text{Hom}}
\newcommand{\id}{\text{id}}
\newcommand{\Tr}{\text{Tr}}

\title{Hierarchical Projection Model of Quantum Measurement:\\Testable Deviations from Standard Theory}
\author{Karol Kowalczyk}
\date{November 9, 2025}

\begin{document}

\maketitle

\begin{abstract}
We propose that quantum measurement can be expressed as hierarchical projection between computational levels in a finite information hierarchy. Each level $n$ corresponds to a Hilbert space of dimension $2^n$. Measurement is modeled as projection $P_Q: \mathcal{H}_j \to \mathcal{H}_i$ from level $j$ to $i < j$, with collapse operators $C_Q: \mathcal{H}_i \to \mathcal{H}_j$ forming an approximate adjunction $(C_Q \dashv P_Q)$. Extending the \emph{Adjoint Projections on Computational Hierarchies} framework to quantum systems, we show that completely positive trace-preserving (CPTP) maps realize $\varepsilon$-adjunctions with deviations quantified by decoherence parameters. The model predicts two testable deviations from standard quantum mechanics: (1) finite measurement delay scaling as $\tau \propto n^2$ for $n$-qubit systems, and (2) small oscillatory corrections to the Born rule from cross-level interference. These effects should be observable in mid-scale systems (10--20 qubits) using current ion-trap and superconducting-qubit technology. We provide detailed experimental protocols, derive thermodynamic implications via a modified Landauer bound, and discuss interpretational consequences. The framework offers concrete, falsifiable predictions distinguishing it from standard quantum theory while remaining agnostic about ontological questions.
\end{abstract}

\noindent\textbf{Keywords:} Quantum measurement, computational hierarchy, CPTP maps, Born rule, decoherence, information theory

\noindent\textbf{Note:} Adjunction in this paper refers to category-theoretic functorial duality $(C_Q \dashv P_Q)$, not Hermitian conjugation $(A^\dagger)$ of operators.

\tableofcontents

\newpage

\section{Glossary of Symbols}
\label{sec:glossary}

For ease of reference, we collect the main notation:

\begin{center}
\begin{tabular}{ll}
\hline
\textbf{Symbol} & \textbf{Meaning} \\
\hline
$\mathcal{H}_n$ & Hilbert space at level $n$ with $\dim \mathcal{H}_n = 2^n$ \\
$\rho$ & Density operator (density matrix) \\
$U_n$ & Unitary operator on $\mathcal{H}_n$ \\
$P_Q$ & Quantum projection operator (CPTP map) \\
$C_Q$ & Quantum collapse operator (embedding) \\
$\eta_Q$ & Unit of quantum adjunction $\id \Rightarrow C_Q \circ P_Q$ \\
$\varepsilon_Q$ & Counit of quantum adjunction $P_Q \circ C_Q \Rightarrow \id$ \\
$\Tr_E$ & Partial trace over environment $E$ \\
$S(\rho)$ & Von Neumann entropy: $-\Tr(\rho \log \rho)$ \\
$D(\rho, \sigma)$ & Trace distance: $\frac{1}{2}\Tr|\rho - \sigma|$ \\
$\tau$ & Measurement delay time \\
$n$ & Hierarchy level (number of qubits) \\
$\varepsilon$ & Approximation error in adjunction \\
$\gamma$ & Decoherence rate \\
$\Hilbfsm$ & Category of finite-dimensional Hilbert spaces \\
\hline
\end{tabular}
\end{center}

\section{Introduction}

\subsection{The measurement problem and information-theoretic approaches}

Quantum mechanics predicts measurement outcomes probabilistically via the Born rule but treats wavefunction collapse as instantaneous and non-dynamical. This ``measurement problem'' has generated numerous interpretations---Copenhagen, Many-Worlds, objective collapse models---each addressing the issue philosophically without providing testable deviations from standard predictions.

Recent information-theoretic approaches (QBism, relational QM, constructor theory) reframe measurement as knowledge update or agent-relative state assignment. While conceptually appealing, these frameworks typically don't predict new measurable phenomena. We take a different approach: treating measurement as \textbf{finite computation} within an explicitly constructed hierarchy of information-processing levels.

\subsection{Hierarchical computation and quantum systems}

The \emph{Adjoint Projections on Computational Hierarchies} framework \cite{kowalczyk2025} formalizes nested computational levels $\{M_n\}$ with state spaces of size $2^n$. Projection operators $P_{j\to i}$ compress information from level $j$ to level $i < j$, while collapse operators $C_{i\to j}$ reconstruct higher-level structure. The pair $(C, P)$ forms an adjunction satisfying category-theoretic identities.

\textbf{Key insight:} Mapping this structure to quantum systems by identifying $n = \log_2(\dim \mathcal{H})$, we interpret measurement as projection between hierarchy levels. Crucially, we assume projection requires \textbf{finite time} proportional to the computational complexity of comparing $2^n$ states.

\subsection{Scope and non-goals}

This framework provides a \emph{formal mathematical structure} for quantum measurement without committing to specific ontological interpretations. We do not claim that:
\begin{itemize}
\item Computational levels are the fundamental constituents of reality
\item Measurement is "really" a computational process
\item Standard quantum mechanics is incorrect
\end{itemize}

Rather, we show that \emph{if} measurement involves hierarchical information processing, then specific testable predictions follow. The framework is agnostic about whether this structure reflects physical reality or is merely a useful mathematical model. Physical validation---or falsification---will determine its empirical status.

\subsection{Main predictions}

Two testable consequences follow:

\begin{enumerate}
\item \textbf{Projection delay:} Measurement time $\tau$ scales as $\tau \propto n^2$ (quadratically with qubit number), contrasting with standard QM ($\tau = 0$, instantaneous) and decoherence theory ($\tau$ independent of $n$ or linear in $n$).

\item \textbf{Born-rule oscillations:} Cross-level interference introduces small periodic corrections: $P(\text{outcome}) = |\alpha|^2 + A\cdot\sin(2\pi\tau/T)$ where $A \approx 1$--5\% and period $T$ depends on level separation.
\end{enumerate}

Both predictions are testable with current technology in 10--20 qubit systems.

\subsection{Critical concerns and limitations}

Before proceeding, we must acknowledge fundamental concerns with this approach:

\begin{concern}[Physical justification]
\label{concern:physical}
\textbf{Why should quantum measurement involve computational hierarchy?} Decoherence theory successfully explains measurement outcomes without invoking computational levels. The connection between quantum projection and computational complexity appears forced rather than derived from fundamental principles. We have no mechanism explaining why nature would implement hierarchical projection.
\end{concern}

\begin{concern}[Arbitrary scaling assumption]
\label{concern:scaling}
The $\tau \propto n^2$ prediction assumes sequential comparison of $2^n$ states. Why not parallel processing ($\tau \propto n$) or tree-based comparison ($\tau \propto n \log n$)? The scaling is stipulated, not derived. Without physical justification for the computational model, the prediction is arbitrary.
\end{concern}

\begin{concern}[Conflict with established theory]
\label{concern:decoherence}
Decoherence theory provides a well-tested, physically motivated account of measurement via environmental entanglement. What advantage does the hierarchical model provide? It adds computational structure without clear physical motivation or superior explanatory power.
\end{concern}

\begin{concern}[Observable deviations should exist]
\label{concern:observable}
Cross-level interference (Section~\ref{sec:born}) predicts 1--5\% deviations from Born rule. Precision tests of quantum mechanics constrain deviations to $<0.01$\% in many systems. If the effect exists at predicted magnitude, it should already be observable---yet it hasn't been reported.
\end{concern}

\begin{concern}[Interpretation ambiguity]
\label{concern:interpretation}
The framework claims to be ``interpretation-neutral'' but makes strong ontological assumptions about computational levels being physically real. This is a disguised interpretation, not a neutral formalism.
\end{concern}

\begin{limitation}[No preliminary data]
\label{lim:data}
This paper presents purely theoretical predictions without pilot experiments, reanalysis of existing data, or even order-of-magnitude feasibility checks. The predictions remain untested speculation.
\end{limitation}

\subsection{Justification for publication despite concerns}

Given these serious issues, why present this framework? Three reasons:

\begin{enumerate}
\item \textbf{Falsifiability:} The predictions are concrete and testable. Falsification would be scientifically valuable, constraining how measurement relates to computation.

\item \textbf{Alternative perspective:} Even if ultimately wrong, exploring computational approaches to measurement may inspire new experimental techniques or theoretical insights.

\item \textbf{Explicit limitations:} By clearly stating weaknesses upfront, we enable informed critique and avoid misleading claims.
\end{enumerate}

The framework should be viewed as \emph{highly speculative} but \emph{rigorously falsifiable}.

\subsection{Relationship to prior work}

Our approach differs from:
\begin{itemize}
\item \textbf{Decoherence theory} \cite{zurek2003,joos2003,schlosshauer2007}: We predict $\tau \propto n^2$, not $\tau \sim$ constant or $\tau \propto n$
\item \textbf{Objective collapse} \cite{penrose1996}: We derive $\tau$ from information processing, not spontaneous localization
\item \textbf{Quantum Darwinism} \cite{zurek2009}: We focus on single-system measurement, not environmental redundancy
\item \textbf{Constructor theory} \cite{deutsch2015}: We provide computational implementation with complexity bounds
\item \textbf{Categorical quantum mechanics} \cite{abramsky2004}: We use adjunctions without dagger compact structure; our focus is computational cost not compositionality
\end{itemize}

\textbf{Novel aspect:} Connecting measurement dynamics to computational complexity via explicit hierarchy levels and adjunction structure---though physical motivation remains unclear (Concern~\ref{concern:physical}).

\subsection{Paper organization}

Section~\ref{sec:framework} reviews the hierarchical framework. Section~\ref{sec:cptp} develops CPTP maps as approximate adjunctions. Section~\ref{sec:delay} derives the $\tau \propto n^2$ scaling. Section~\ref{sec:born} analyzes Born-rule corrections. Section~\ref{sec:experiments} presents experimental protocols. Section~\ref{sec:thermo} discusses thermodynamics. Section~\ref{sec:interpretation} addresses interpretation. Section~\ref{sec:falsification} states limitations and falsifiability criteria. Section~\ref{sec:conclusion} concludes.

\section{Framework Overview: Computational Hierarchies and Quantum Systems}
\label{sec:framework}

\subsection{Review: Finite computational machines}

From \cite{kowalczyk2025}, a computational hierarchy $\{M_n\}_{n\in\mathbb{N}}$ consists of finite machines $M_n = (S_n, f_n, \pi_n)$ where:
\begin{itemize}
\item State space $S_n$ has cardinality $|S_n| = 2^n$
\item Transition function $f_n: S_n \to S_n$ is deterministic
\item Probability distribution $\pi_n: S_n \to [0,1]$ satisfies $\sum_s \pi_n(s) = 1$
\end{itemize}

Information capacity is $I_n = n$ bits. Levels connect via:
\begin{itemize}
\item \textbf{Embeddings} $\sigma_{i\to j}: S_i \hookrightarrow S_j$ (injective, structure-preserving)
\item \textbf{Projections} $P_{j\to i}: S_j \to S_i$ (surjective, entropy-minimizing)
\item \textbf{Collapses} $C_{i\to j}: S_i \to S_j$ (injective, left adjoint to $P$)
\end{itemize}

The pair $(C, P)$ forms a category-theoretic adjunction with unit $\eta: \id \Rightarrow C \circ P$ and counit $\varepsilon: P \circ C \Rightarrow \id$.

\subsection{Quantum analog: Hilbert space hierarchy}

\begin{definition}[Quantum hierarchy]
A quantum hierarchy $\{\mathcal{H}_n\}_{n\in\mathbb{N}}$ consists of finite-dimensional Hilbert spaces with $\dim \mathcal{H}_n = 2^n$, equipped with:
\begin{itemize}
\item Density operators $\rho_n$ on $\mathcal{H}_n$
\item Unitary evolution $U_n: \mathcal{H}_n \to \mathcal{H}_n$
\item CPTP maps $\mathcal{E}_n: \mathcal{L}(\mathcal{H}_n) \to \mathcal{L}(\mathcal{H}_n)$ on the space of linear operators
\end{itemize}
\end{definition}

For $n$-qubit systems, $\dim \mathcal{H}_n = 2^n$ corresponds directly to level $n$ in the computational hierarchy.

\subsection{Category $\Hilbfsm$: Morphisms in finite quantum hierarchy}

\begin{definition}[Category $\Hilbfsm$]
The category $\Hilbfsm$ has:
\begin{itemize}
\item \textbf{Objects:} Finite-dimensional Hilbert spaces $\mathcal{H}_n$ with $\dim \mathcal{H}_n = 2^n$
\item \textbf{Morphisms:} CPTP maps $\mathcal{E}: \mathcal{L}(\mathcal{H}_i) \to \mathcal{L}(\mathcal{H}_j)$ that preserve density operator evolution: $\mathcal{E}(U_i \rho U_i^\dagger) = U_j \mathcal{E}(\rho) U_j^\dagger$ for appropriate unitaries
\item \textbf{Composition:} Standard composition of CPTP maps
\item \textbf{Identities:} $\id_{\mathcal{H}_n}$ is the identity CPTP map
\end{itemize}
\end{definition}

This parallels the category $\Fsm$ for classical computational hierarchies, with CPTP maps replacing transition-preserving functions.

\subsection{Quantum projection and collapse operators}

\begin{definition}[Quantum projection $P_Q$]
\label{def:projection_q}
For $i < j$, the quantum projection $P_Q: \mathcal{L}(\mathcal{H}_j) \to \mathcal{L}(\mathcal{H}_i)$ is given by the partial trace:
\begin{equation}
P_Q(\rho_{ij}) = \Tr_E(\rho_{ij})
\end{equation}
where $\mathcal{H}_j \cong \mathcal{H}_i \otimes \mathcal{H}_E$ with $\dim \mathcal{H}_E = 2^{j-i}$.
\end{definition}

\begin{remark}[Normalization]
We assume $\Tr_E(\rho_{ij})$ is automatically normalized to trace 1, which holds when the environment factor $\mathcal{H}_E$ is properly separable. For mixed states, explicit renormalization may be required: $P_Q(\rho) = \Tr_E(\rho) / \Tr(\Tr_E(\rho))$.
\end{remark}

\begin{definition}[Quantum collapse $C_Q$]
\label{def:collapse_q}
For $i < j$, the quantum collapse $C_Q: \mathcal{L}(\mathcal{H}_i) \to \mathcal{L}(\mathcal{H}_j)$ is given by:
\begin{equation}
C_Q(\rho_i) = \rho_i \otimes \rho_E
\end{equation}
where $\rho_E$ is a fixed state (typically maximally mixed) on $\mathcal{H}_E$.
\end{definition}

\subsection{Adjunction structure}

\begin{lemma}[Triangle identities for $\Hilbfsm$]
\label{lem:triangle}
The quantum projection $P_Q$ and collapse $C_Q$ satisfy the triangle identities:
\begin{enumerate}
\item $(\varepsilon_Q P_Q) \circ (P_Q \eta_Q) = \id_{P_Q}$
\item $(C_Q \varepsilon_Q) \circ (\eta_Q C_Q) = \id_{C_Q}$
\end{enumerate}
where $\eta_Q: \id \Rightarrow C_Q \circ P_Q$ and $\varepsilon_Q: P_Q \circ C_Q \Rightarrow \id$.
\end{lemma}

\begin{proof}
We verify each identity:

\textbf{(1) First triangle identity:}
For $\rho \in \mathcal{L}(\mathcal{H}_j)$, we have:
\begin{align*}
P_Q(\rho) &= \Tr_E(\rho) \\
\eta_Q(P_Q(\rho)) &= C_Q(P_Q(\rho)) = P_Q(\rho) \otimes \rho_E \\
P_Q(\eta_Q(P_Q(\rho))) &= P_Q(P_Q(\rho) \otimes \rho_E) = \Tr_E(P_Q(\rho) \otimes \rho_E) = P_Q(\rho) \cdot \Tr(\rho_E) = P_Q(\rho)
\end{align*}
since $\Tr(\rho_E) = 1$. Thus $\varepsilon_Q(P_Q \circ \eta_Q) = \id$.

\textbf{(2) Second triangle identity:}
For $\sigma \in \mathcal{L}(\mathcal{H}_i)$, we have:
\begin{align*}
C_Q(\sigma) &= \sigma \otimes \rho_E \\
P_Q(C_Q(\sigma)) &= \Tr_E(\sigma \otimes \rho_E) = \sigma \cdot \Tr(\rho_E) = \sigma \\
\eta_Q(\sigma) &= C_Q(P_Q(C_Q(\sigma))) = C_Q(\sigma)
\end{align*}
Thus $C_Q \circ \varepsilon_Q = \id$. \checkmark
\end{proof}

\begin{theorem}[Exact quantum adjunction]
\label{thm:exact_quantum}
For noiseless CPTP maps $P_Q$ and $C_Q$ defined above, $(C_Q \dashv P_Q)$ forms an exact adjunction in $\Hilbfsm$.
\end{theorem}

\section{CPTP Maps and Approximate Adjunctions}
\label{sec:cptp}

\subsection{Decoherence and approximate adjunction}

In realistic quantum systems, decoherence causes deviations from exact adjunction. The approximation error $\varepsilon$ is related to the decoherence rate $\gamma$ and evolution time $t$:

\begin{theorem}[$\varepsilon$-adjunction with decoherence]
\label{thm:approx_adj}
For a quantum system with decoherence rate $\gamma$, the CPTP maps $P_Q$ and $C_Q$ satisfy $\varepsilon$-adjunction:
\begin{equation}
\|(C_Q \circ P_Q)(\rho) - \rho\| \leq \varepsilon
\end{equation}
where $\varepsilon \approx e^{-\gamma t}$ and $\|\cdot\|$ is the trace norm.
\end{theorem}

\begin{remark}[Physical interpretation]
For open quantum systems, $\varepsilon$ quantifies the trace distance between the ideal adjoint evolution and the actual decoherent evolution. Typical decoherence times for superconducting qubits are $T_2 \approx 10$--$100$ $\mu$s, giving $\gamma \approx 10^4$--$10^5$ Hz. For measurement times $\tau \sim 1$ $\mu$s, we expect $\varepsilon \approx 0.01$--$0.1$.
\end{remark}

\subsection{Information loss and von Neumann entropy}

The information loss during projection is quantified by:
\begin{equation}
\Delta S = S(\rho_j) - S(\rho_i) = S(\rho_j) - S(P_Q(\rho_j))
\end{equation}
where $S(\rho) = -\Tr(\rho \log \rho)$ is the von Neumann entropy. For pure states, $S(\rho_j) = 0$ and $S(P_Q(\rho_j)) \geq 0$, consistent with entropy increase from tracing out the environment.

\section{Measurement Delay and $n^2$ Scaling}
\label{sec:delay}

\subsection{Computational complexity of projection}

The projection $P_Q$ requires comparing density operators across $2^n$ basis states. Assuming sequential pairwise comparisons, the time complexity is:
\begin{equation}
\tau(n) \propto \frac{(2^n)^2}{2} = 2^{2n-1} \propto n^2 \quad \text{(for small } n \text{)}
\end{equation}

\textbf{Note:} This scaling is \emph{stipulated}, not derived from fundamental physics (Concern~\ref{concern:scaling}). Alternative models (parallel, tree-based) would give different scalings.

\subsection{Experimental predictions}

For $n$-qubit systems:
\begin{itemize}
\item $n = 5$: $\tau \approx 25 \tau_0$
\item $n = 10$: $\tau \approx 100 \tau_0$
\item $n = 15$: $\tau \approx 225 \tau_0$
\item $n = 20$: $\tau \approx 400 \tau_0$
\end{itemize}
where $\tau_0$ is a system-dependent time constant (estimated $\sim 100$ ns for superconducting qubits).

These predictions can be tested by measuring single-qubit vs. multi-qubit measurement times with controlled decoherence.

\section{Born Rule Oscillations}
\label{sec:born}

\subsection{Cross-level interference}

When measurement involves projection between non-adjacent levels (e.g., $j \to i$ with $j - i > 1$), intermediate levels can interfere, producing oscillatory corrections:
\begin{equation}
P(\text{outcome}) = |\alpha|^2 + A \cdot \sin(2\pi \tau / T) + O(A^2)
\end{equation}
where:
\begin{itemize}
\item $A \approx 0.01$--$0.05$ (1--5\% amplitude)
\item $T = 2\pi\hbar / \Delta E$ with $\Delta E$ the energy scale of intermediate levels
\item $\tau$ is the measurement time
\end{itemize}

\textbf{Critical issue:} These deviations should already be observable if they exist at the predicted magnitude (Concern~\ref{concern:observable}).

\section{Experimental Protocols}
\label{sec:experiments}

\subsection{Protocol 1: Measurement time vs. qubit number}

\begin{enumerate}
\item Prepare $n$-qubit systems ($n = 5, 10, 15, 20$) in superposition states
\item Perform projective measurements with high-precision timing (ps resolution)
\item Measure time from measurement initiation to outcome registration
\item Plot $\tau$ vs. $n$ and fit to power law: $\tau \propto n^\beta$
\item Compare $\beta$ to predicted value (2) and alternatives (0, 1, $\log n$)
\end{enumerate}

\subsection{Protocol 2: Born rule precision tests}

\begin{enumerate}
\item Prepare qubits in known superposition: $|\psi\rangle = \alpha|0\rangle + \beta|1\rangle$
\item Measure outcome probabilities over $10^6$ trials
\item Fit to $P(0) = |\alpha|^2 + A\sin(2\pi t/T)$
\item Test null hypothesis: $A = 0$ (standard Born rule)
\item Report upper bound on $A$ at 95\% confidence
\end{enumerate}

\section{Thermodynamic Implications}
\label{sec:thermo}

\subsection{Modified Landauer bound}

The energy cost of measurement is:
\begin{equation}
E_{\text{measure}} = \kappa k_B T \ln 2 \cdot \Delta S
\end{equation}
where $\Delta S = j - i$ is the entropy loss in bits, and $\kappa$ is an inefficiency factor.

For quantum measurements, $\kappa \gg 1$ due to decoherence overhead, consistent with experimental observations \cite{landauer1961,bennett1982}.

\section{Interpretational Consequences}
\label{sec:interpretation}

\subsection{Ontological commitments}

The framework claims to be ``interpretation-neutral'' but implicitly assumes:
\begin{itemize}
\item Computational levels are physically real (ontological)
\item Measurement is an objective process with definite duration (ontological)
\item Hierarchy structure exists independent of observers (ontological)
\end{itemize}

This contradicts instrumentalist/epistemic interpretations (Copenhagen, QBism) where measurement is knowledge update, not physical projection.

\begin{concern}[Hidden interpretation]
The hierarchical framework is not interpretation-neutral. It presupposes realist ontology where computational structure is fundamental. This should be acknowledged explicitly rather than claimed as neutral formalism.
\end{concern}

\subsection{Relation to Many-Worlds}

In Many-Worlds, measurement creates branching without collapse. Hierarchical projection requires actual collapse (information loss). These are incompatible.

\subsection{Relation to decoherence}

Decoherence explains \emph{apparent} collapse via entanglement with environment. Hierarchical model adds computational layer without clear advantage (Concern~\ref{concern:decoherence}).

\section{Limitations and Falsifiability}
\label{sec:falsification}

\subsection{Summary of critical concerns}

\begin{enumerate}
\item \textbf{Physical justification lacking} (Concern~\ref{concern:physical}): No mechanism for why nature implements hierarchy
\item \textbf{Arbitrary scaling} (Concern~\ref{concern:scaling}): $\tau \propto n^2$ stipulated, not derived
\item \textbf{Conflicts with decoherence} (Concern~\ref{concern:decoherence}): Existing theory more parsimonious
\item \textbf{Observable constraints} (Concern~\ref{concern:observable}): Born rule tested to $<0.01$\%, yet we predict 1--5\% effects
\item \textbf{Interpretation ambiguity} (Concern~\ref{concern:interpretation}): Claims neutrality while making ontological commitments
\item \textbf{No preliminary data} (Limitation~\ref{lim:data}): Pure speculation without feasibility tests
\end{enumerate}

\subsection{Falsification criteria}

The framework is falsified if:

\begin{enumerate}
\item \textbf{Measurement time is constant or linear:} $\tau(n) = \tau_0$ or $\tau(n) \propto n$ with $p < 0.01$
\item \textbf{Born rule holds exactly:} Oscillation amplitude $A < 0.001$ (below current experimental precision)
\item \textbf{Deviations have wrong structure:} Effects observed but with different scaling (e.g., $\tau \propto e^n$)
\end{enumerate}

\subsection{Value if falsified}

Even if wrong, testing these predictions:
\begin{itemize}
\item Constrains relationship between measurement and computation
\item Motivates precision timing experiments
\item Rules out entire class of hierarchical models
\end{itemize}

\subsection{Implications if confirmed}

If predictions hold:
\begin{itemize}
\item Measurement has finite computational cost
\item Nature implements hierarchical information processing
\item Standard QM requires modification
\end{itemize}

But confirmation would require:
\begin{enumerate}
\item Multiple independent experimental groups
\item Systematic exclusion of artifacts
\item Theoretical explanation for \emph{why} hierarchy exists
\end{enumerate}

\section{Conclusion}
\label{sec:conclusion}

We have shown that quantum measurement can be expressed within a hierarchical projection framework with finite computational cost, yielding testable predictions: $\tau \propto n^2$ and small Born rule deviations.

\textbf{Critical assessment:}
\begin{itemize}
\item \textbf{Strengths:} Concrete falsifiable predictions, explicit mathematical framework via adjunction $(C_Q \dashv P_Q)$, rigorous categorical structure
\item \textbf{Weaknesses:} Lacks physical justification, arbitrary assumptions about scaling, conflicts with established decoherence theory
\end{itemize}

This work should be viewed as \textbf{highly speculative} but \textbf{rigorously testable}. The predictions are likely false, but testing them constrains how computation relates to quantum mechanics. Even negative results advance our understanding.

The framework reframes collapse from axiom to algorithm, but whether nature actually performs this algorithm remains an open---and skepticism-inducing---question.

\begin{thebibliography}{10}

\bibitem{abramsky2004}
Abramsky, S., \& Coecke, B. (2004).
\newblock A categorical semantics of quantum protocols.
\newblock \emph{Proceedings of LICS}, 415--425.

\bibitem{bennett1982}
Bennett, C. H. (1982).
\newblock The thermodynamics of computation---a review.
\newblock \emph{International Journal of Theoretical Physics}, 21(12), 905--940.

\bibitem{deutsch2015}
Deutsch, D., \& Marletto, C. (2015).
\newblock Constructor theory of information.
\newblock \emph{Proceedings of the Royal Society A}, 471(2174), 20140540.

\bibitem{fuchs1999}
Fuchs, C. A., \& Van De Graaf, J. (1999).
\newblock Cryptographic distinguishability measures for quantum-mechanical states.
\newblock \emph{IEEE Transactions on Information Theory}, 45(4), 1216--1227.

\bibitem{joos2003}
Joos, E., et al. (2003).
\newblock \emph{Decoherence and the Appearance of a Classical World in Quantum Theory} (2nd ed.).
\newblock Springer.

\bibitem{kowalczyk2025}
Kowalczyk, K. (2025).
\newblock Adjoint projections on computational hierarchies: A metric framework.
\newblock \emph{Manuscript in preparation}.

\bibitem{landauer1961}
Landauer, R. (1961).
\newblock Irreversibility and heat generation in the computing process.
\newblock \emph{IBM Journal of Research and Development}, 5(3), 183--191.

\bibitem{nielsen2010}
Nielsen, M. A., \& Chuang, I. L. (2010).
\newblock \emph{Quantum Computation and Quantum Information} (10th Anniversary Edition).
\newblock Cambridge University Press.

\bibitem{penrose1996}
Penrose, R. (1996).
\newblock On gravity's role in quantum state reduction.
\newblock \emph{General Relativity and Gravitation}, 28(5), 581--600.

\bibitem{schlosshauer2007}
Schlosshauer, M. (2007).
\newblock \emph{Decoherence and the Quantum-to-Classical Transition}.
\newblock Springer.

\bibitem{zurek2003}
Zurek, W. H. (2003).
\newblock Decoherence, einselection, and the quantum origins of the classical.
\newblock \emph{Reviews of Modern Physics}, 75(3), 715--775.

\bibitem{zurek2009}
Zurek, W. H. (2009).
\newblock Quantum Darwinism.
\newblock \emph{Nature Physics}, 5(3), 181--188.

\end{thebibliography}

\end{document}
